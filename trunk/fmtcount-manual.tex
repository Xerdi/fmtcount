\documentclass{nlctdoc}

\usepackage{color}
\usepackage[utf8]{inputenc}
\usepackage[french,english]{babel}
\usepackage{amsmath,amssymb}
\usepackage{tabularx,supertabular,multirow}
\usepackage[T1]{fontenc}
\usepackage{cmap}
\usepackage{fmtcount}% needed for \fc@iterate@on@languages
\newcommand*\uref[1]{\href{#1}{#1}}
\DeclareMathOperator{\intv}{..}
\newcommand*\interface{}
\makeatletter
\newcommand*{\LanguagePackageOptionList}{{%
 \def\@tempf##1{\texttt{##1}}% how to format each option
 % we don't use \cs{newtoks} because anyway this is within a group
 \toks0{}% token in which we accumulate result
 %  \toks1 token in which we place next dialect
 \count0 0 % to distinguish first and second iteration, we need to do tricks because we want to insert `and'
           % before the last item in the list.
 \def\@tempb##1{%
   \ifnum\count0=0 %
   \else\ifnum\count0=1 %
     \toks0\toks1 %
   \else
     \edef\@tempa{\the\toks0, \the\toks1}%
     \toks0\expandafter{\@tempa}%
   \fi\fi
   \toks1\expandafter{\@tempf{##1}}%
   \advance\count0 by 1 %
 }%
 \fc@iterate@on@languages\@tempb
 \edef\@tempa{\the\toks0, and \the\toks1}%
 \expandafter
}\@tempa}
\makeatother

\usepackage[colorlinks,
            bookmarks,
            bookmarksopen,
            pdfauthor={Nicola L.C. Talbot},
            pdftitle={fmtcount.sty: Displaying the Values of LaTeX Counters},
            pdfkeywords={LaTeX,counter}]{hyperref}

\CheckSum{10802}

\doxitem{Option}{option}{options}

\newlength{\tabcolwidth}
\newlength{\coliwidth}
\newcommand*\cnab{\centering\arraybackslash}
\newcommand*\ColIRI[1]{}
\csname @ifpackageloaded\endcsname{tex4ht}{%
\newcommand*\ColIRII[1]{\parbox{\coliwidth}{\cnab #1}}
}{%
\newcommand*\ColIRII[1]{\multirow{-2}{\coliwidth}{\cnab #1}}
}

\begin{document}
\RecordChanges

 \title{fmtcount.sty: Displaying the Values of 
\LaTeX\ Counters}
 \author{Nicola L.C. Talbot\\
 \url{www.dickimaw-books.com}
 \and
 Vincent Bela\"iche}
 \date{%%DATE%% (version %%VERSION%%)\relax
}
 \maketitle
 \tableofcontents
 \section{Introduction}

The \styfmt{fmtcount} package provides commands to display
the values of \LaTeX\ counters in a variety of formats. It also
provides equivalent commands for actual numbers rather than 
counter names. Limited multilingual support is available. 
Currently, there is only support for English, French (including
Belgian and Swiss variations), Spanish, Portuguese, German and 
Italian.

\section{Available Commands}

The commands can be divided into two categories: those that
take the name of a counter as the argument, and those
that take a number as the argument.

\begin{definition}[\DescribeMacro{\ordinal}]
\cs{ordinal}\marg{counter}\oarg{gender}
\end{definition}
This will print the value of a \LaTeX\ counter \meta{counter} as an 
ordinal, where the macro
\begin{definition}[\DescribeMacro{\fmtord}]
\cs{fmtord}\marg{text}
\end{definition}
is used to format the st, nd, rd, th bit.  By default the ordinal is
formatted as a superscript, if the package option \pkgopt{level} is
used, it is level with the text.  For example, if the current section
is \arabic{section}, then \verb"\ordinal{section}" will produce the output:
\ordinal{section}. Note that the optional argument \meta{gender} 
occurs \emph{at the end}. This argument may only take one of
the following values: \texttt{m} (masculine), \texttt{f}
(feminine) or \texttt{n} (neuter.) If \meta{gender} is omitted, 
or if the given gender has no meaning in the current language, 
\texttt{m} is assumed.

\textbf{Notes:} 
\begin{enumerate}
\item the \cls{memoir} class also defines a command called
\cs{ordinal} which takes a number as an argument instead of a
counter. In order to overcome this incompatiblity, if you want
to use the \styfmt{fmtcount} package with the \cls{memoir} class 
you should use
\begin{definition}[\DescribeMacro{\FCordinal}]
\cs{FCordinal}
\end{definition}
to access \styfmt{fmtcount}'s version of \cs{ordinal}, and use
\cs{ordinal} to use \cls{memoir}'s version of that command.
\item When the \oarg{gender} optional argument is omitted, no ignoring of spaces
  following the final argument occurs. So both
  \verb"\ordinal{section}"\textvisiblespace\verb"!"  and
  \verb"\ordinal{section}[m]"\textvisiblespace\verb"!" will produce:
  \ordinal{section}\textvisiblespace!, where \textvisiblespace\ denotes a space.  See
  \S~\ref{sec:tailing-oarg}.
\end{enumerate}

The commands below only work for numbers in the range 0~to~99999.
\begin{definition}[\DescribeMacro{\ordinalnum}]
\cs{ordinalnum}\marg{n}\oarg{gender}
\end{definition}
This is like \cs{ordinal} but takes an actual number rather than a counter as the argument. For example:
\verb"\ordinalnum{"\texttt{\arabic{section}}\verb"}" will produce: \ordinal{section}.

\begin{definition}[\DescribeMacro{\numberstring}]
\cs{numberstring}\marg{counter}\oarg{gender}
\end{definition}
This will print the value of \meta{counter} as text.  E.g.\
\verb"\numberstring{section}" will produce: three. The optional
argument is the same as that for \cs{ordinal}.

\begin{definition}[\DescribeMacro{\Numberstring}]
\cs{Numberstring}\marg{counter}\oarg{gender}
\end{definition}
This does the same as \cs{numberstring}, but with initial letters in
uppercase.  For example, \verb"\Numberstring{section}" will produce:
\Numberstring{section}.

\begin{definition}[\DescribeMacro{\NUMBERstring}]
\cs{NUMBERstring}\marg{counter}\oarg{gender}
\end{definition}
This does the same as \cs{numberstring}, but converts the string to
upper case. Note that
\verb"\MakeUppercase{\NUMBERstring{"\meta{counter}\verb'}}' doesn't
work, due to the way that \cs{MakeUppercase} expands its
argument\footnote{See all the various postings to
\texttt{comp.text.tex} about \cs{MakeUppercase}}.

\begin{definition}[\DescribeMacro{\numberstringnum}]
\cs{numberstringnum}\marg{n}\oarg{gender}
\end{definition}
\begin{definition}[\DescribeMacro{\Numberstringnum}]
\cs{Numberstringnum}\marg{n}\oarg{gender}
\end{definition}
\begin{definition}[\DescribeMacro{\NUMBERstringnum}]
\cs{NUMBERstringnum}\marg{n}\oarg{gender}
\end{definition}
Theses macros  work like 
\cs{numberstring}, \cs{Numberstring} and \cs{NUMBERstring}, 
respectively, but take an actual number
rather than a counter as the argument. For example:
\verb'\Numberstringnum{105}' will produce: One Hundred and Five.

\begin{definition}[\DescribeMacro{\ordinalstring}]
\cs{ordinalstring}\marg{counter}\oarg{gender}
\end{definition}
This will print the value of \meta{counter} as a textual ordinal.
E.g.\ \verb"\ordinalstring{section}" will produce: third. The
optional argument is the same as that for \cs{ordinal}.

\begin{definition}[\DescribeMacro{\Ordinalstring}]
\cs{Ordinalstring}\marg{counter}\oarg{gender}
\end{definition}
This does the same as \cs{ordinalstring}, but with initial letters in
uppercase.  For example, \verb"\Ordinalstring{section}" will produce:
\Ordinalstring{section}.

\begin{definition}[\DescribeMacro{\ORDINALstring}]
\cs{ORDINALstring}\marg{counter}\oarg{gender}
\end{definition}
This does the same as \cs{ordinalstring}, but with all words in upper
case (see previous note about \cs{MakeUppercase}).

\begin{definition}[\DescribeMacro{\ordinalstringnum}]
\cs{ordinalstringnum}\marg{n}\oarg{gender}
\end{definition}
\begin{definition}[\DescribeMacro{\Ordinalstringnum}]
\cs{Ordinalstringnum}\marg{n}\oarg{gender}
\end{definition}
\begin{definition}[\DescribeMacro{\ORDINALstringnum}]
\cs{ORDINALstringnum}\marg{n}\oarg{gender}
\end{definition}
These macros work like \cs{ordinalstring}, \cs{Ordinalstring} and
\cs{ORDINALstring}, respectively, but take an actual number rather than a
counter as the argument. For example,
\verb"\ordinalstringnum{"\texttt{\arabic{section}}\verb"}" will produce:
\ordinalstring{section}.

\changes{v.1.09}{21 Apr 2007}{store facility added}
As from version 1.09, textual representations can be stored for
later use. This overcomes the problems encountered when you
attempt to use one of the above commands in \cs{edef}.

Each of the following commands takes a label as the first argument,
the other arguments are as the analogous commands above. These
commands do not display anything, but store the textual 
representation. This can later be retrieved using

\begin{definition}[\DescribeMacro{\FMCuse}]
\cs{FMCuse}\marg{label}
\end{definition}

\begin{important}
Note: with \cs{storeordinal} and \cs{storeordinalnum}, the 
only bit that doesn't get expanded is \cs{fmtord}. So, for 
example, \verb"\storeordinalnum{mylabel}{3}" will be stored as
\verb"3\relax \fmtord{rd}".
\end{important}

\begin{definition}[\DescribeMacro{\storeordinal}]
\cs{storeordinal}\marg{label}\marg{counter}\oarg{gender}
\end{definition}

\begin{definition}[\DescribeMacro{\storeordinalstring}]
\cs{storeordinalstring}\marg{label}\marg{counter}\oarg{gender}
\end{definition}

\begin{definition}[\DescribeMacro{\storeOrdinalstring}]
\cs{storeOrdinalstring}\marg{label}\marg{counter}\oarg{gender}
\end{definition}


\begin{definition}[\DescribeMacro{\storeORDINALstring}]
\cs{storeORDINALstring}\marg{label}\marg{counter}\oarg{gender}
\end{definition}

\begin{definition}[\DescribeMacro{\storenumberstring}]
\cs{storenumberstring}\marg{label}\marg{counter}\oarg{gender}
\end{definition}

\begin{definition}[\DescribeMacro{\storeNumberstring}]
\cs{storeNumberstring}\marg{label}\marg{counter}\oarg{gender}
\end{definition}

\begin{definition}[\DescribeMacro{\storeNUMBERstring}]
\cs{storeNUMBERstring}\marg{label}\marg{counter}\oarg{gender}
\end{definition}

\begin{definition}[\DescribeMacro{\storeordinalnum}]
\cs{storeordinalnum}\marg{label}\marg{number}\oarg{gender}
\end{definition}

\begin{definition}[\DescribeMacro{\storeordinalstringnum}]
\cs{storeordinalstring}\marg{label}\marg{number}\oarg{gender}
\end{definition}

\begin{definition}[\DescribeMacro{\storeOrdinalstringnum}]
\cs{storeOrdinalstringnum}\marg{label}\marg{number}\oarg{gender}
\end{definition}

\begin{definition}[\DescribeMacro{\storeORDINALstringnum}]
\cs{storeORDINALstringnum}\marg{label}\marg{number}\oarg{gender}
\end{definition}

\begin{definition}[\DescribeMacro{\storenumberstringnum}]
\cs{storenumberstring}\marg{label}\marg{number}\oarg{gender}
\end{definition}

\begin{definition}[\DescribeMacro{\storeNumberstringnum}]
\cs{storeNumberstring}\marg{label}\marg{number}\oarg{gender}
\end{definition}

\begin{definition}[\DescribeMacro{\storeNUMBERstringnum}]
\cs{storeNUMBERstring}\marg{label}\marg{number}\oarg{gender}
\end{definition}

\begin{definition}[\DescribeMacro{\binary}]
\cs{binary}\marg{counter}
\end{definition}
This will print the value of \meta{counter} as a binary number.  E.g.\
\verb"\binary{section}" will produce: \binary{section}. The declaration
\begin{definition}[\DescribeMacro{\padzeroes}]
\cs{padzeroes}\oarg{n}
\end{definition}
will ensure numbers are written to \meta{n} digits, padding with zeroes if
necessary. E.g.\ \verb"\padzeroes"\discretionary{}{}{}\verb"[8]\binary{section}"
will produce: \padzeroes[8]\binary{section}. The default value for \meta{n} is
17.

\begin{definition}[\DescribeMacro{\binarynum}]
\cs{binary}\marg{n}
\end{definition}
This is like \cs{binary} but takes an actual number rather than a
counter as the argument. For example: \verb"\binarynum{5}" will
produce: 101.

The octal commands only work for values in the range 0~to~32768.
\begin{definition}[\DescribeMacro{\octal}]
\cs{octal}\marg{counter}
\end{definition}
This will print the value of \meta{counter} as an octal number.  For
example, if you have a counter called, say \texttt{mycounter}, and
you set the value to 125, then \verb"\octal{mycounter}" will produce:
177.  Again, the number will be padded with zeroes if necessary,
depending on whether \cs{padzeroes} has been used.

\begin{definition}[\DescribeMacro{\octalnum}]
\cs{octalnum}\marg{n}
\end{definition}
This is like \cs{octal} but takes an actual number rather than a
counter as the argument. For example: \verb"\octalnum{125}" will
produce: 177.

\begin{definition}[\DescribeMacro{\hexadecimal}]
\cs{hexadecimal}\marg{counter}
\end{definition}
This will print the value of \meta{counter} as a hexadecimal number.
Going back to the counter used in the previous example,
\verb"\hexadecimal{mycounter}" will produce: 7d. Again, the number
will be padded with zeroes if necessary, depending on whether
\cs{padzeroes} has been used.

\begin{definition}[\DescribeMacro{\HEXADecimal}]
\cs{HEXADecimal}\marg{counter}
\end{definition}
This does the same thing, but uses uppercase characters, e.g.\
\verb"\HEXADecimal{mycounter}" will produce: 7D.

\DescribeMacro{\Hexadecimal}The macro \cs{Hexadecimal} is a deprecated alias of
\cs{HEXADecimal}. Its name was confusing so it was
changed. See~\ref{sec:macro-naming}.


\begin{definition}[\DescribeMacro{\hexadecimalnum}]
\cs{hexadecimalnum}\marg{n}
\end{definition}

\begin{definition}[\DescribeMacro{\HEXADecimalnum}]
\cs{HEXADecimalnum}\marg{n}
\end{definition}
These are like \cs{hexadecimal} and \cs{Hexadecimal}
but take an actual number rather than a counter as the
argument. For example: \verb"\hexadecimalnum{125}" will
produce: 7d, and \verb"\HEXADecimalnum{125}" will 
produce: 7D.

\DescribeMacro{\Hexadecimalnum}The macro \cs{Hexadecimalnum} is a deprecated
alias of \cs{HEXADecimalnum}. Its name was confusing so it was changed.
See~\ref{sec:macro-naming}.


\begin{definition}[\DescribeMacro{\decimal}]
\cs{decimal}\marg{counter}
\end{definition}
This is similar to \cs{arabic} but the number can be padded with zeroes
depending on whether \cs{padzeroes} has been used.  For example:
\verb"\padzeroes[8]\decimal{section}" will produce:
\padzeroes[8]\decimal{section} still assuming current section is
section~\arabic{section}.

\begin{definition}[\DescribeMacro{\decimalnum}]
\cs{decimalnum}\marg{n}
\end{definition}
This is like \cs{decimal} but takes an actual number rather than a
counter as the argument. For example:
\verb"\padzeroes[8]\decimalnum{5}" will produce: 00000005.

\begin{definition}[\DescribeMacro{\aaalph}]
\cs{aaalph}\marg{counter}
\end{definition}
This will print the value of \meta{counter} as: a b \ldots\ z aa bb \ldots\ zz
etc.  For example, \verb"\aaalpha"\discretionary{}{}{}\verb"{mycounter}" will
produce: uuuuu if \texttt{mycounter} is set to 125.

\begin{definition}[\DescribeMacro{\AAAlph}]
\cs{AAAlph}\marg{counter}
\end{definition}
This does the same thing, but uses uppercase characters, e.g.\
\verb"\AAAlph{mycounter}" will produce: UUUUU.

\begin{definition}[\DescribeMacro{\aaalphnum}]
\cs{aaalphnum}\marg{n}
\end{definition}

\begin{definition}[\DescribeMacro{\AAAlphnum}]
\cs{AAAlphnum}\marg{n}
\end{definition}
These macros are like \cs{aaalph} and \cs{AAAlph}
but take an actual number rather than a counter as the
argument. For example: \verb"\aaalphnum{125}" will
produce: uuuuu, and \verb"\AAAlphnum{125}" will 
produce: UUUUU.

The abalph commands described below only work for values in the
range 0~to~17576.
\begin{definition}[\DescribeMacro{\abalph}]
\cs{abalph}\marg{counter}
\end{definition}
This will print the value of \meta{counter} as: a b \ldots\ z aa ab
\ldots\ az etc.  For example, \verb"\abalpha{mycounter}" will
produce: du if \texttt{mycounter} is set to 125.

\begin{definition}[\DescribeMacro{\ABAlph}]
\cs{ABAlph}\marg{counter}
\end{definition}
This does the same thing, but uses uppercase characters, e.g.\
\verb"\ABAlph{mycounter}" will produce: DU.

\begin{definition}[\DescribeMacro{\abalphnum}]
\cs{abalphnum}\marg{n}
\end{definition}

\begin{definition}[\DescribeMacro{\ABAlphnum}]
\cs{ABAlphnum}\marg{n}
\end{definition}
These macros are like \cs{abalph} and \cs{ABAlph}
but take an actual number rather than a counter as the
argument. For example: \verb"\abalphnum{125}" will
produce: du, and \verb"\ABAlphnum{125}" will 
produce: DU.

\section{Package Options}
\label{sec:package-options}

The following options can be passed to this package:

\begin{description}
\item[\meta{dialect}] load language \meta{dialect}, supported \meta{dialect} are the same as passed to
  \cs{FCloadlang}, see~\ref{sec:multilingual-support}
\item[\pkgopt{raise}] make ordinal st,nd,rd,th appear as superscript
\item[\pkgopt{level}] make ordinal st,nd,rd,th appear level with rest of text
\end{description}


\noindent Options \pkgopt{raise} and \pkgopt{level} can also be set using the command:

\begin{definition}[\DescribeMacro{\fmtcountsetoptions}]
\cs{fmtcountsetoptions}\verb"{fmtord="\meta{type}\verb'}'
\end{definition}
where \meta{type} is either \texttt{level} or \texttt{raise}. Since version~3.01 of \sty{fmtcount}, it is also
possible to set \meta{type} on a language by language basis, see~\S~\ref{sec:multilingual-support}.

\section{Multilingual Support}
\label{sec:multilingual-support}

Version 1.02 of the \sty{fmtcount} package now has
limited multilingual support.  The following languages are
implemented: English, Spanish, Portuguese, French, French (Swiss)
and French (Belgian). German support was added in version 
1.1.\footnote{Thanks to K. H. Fricke for supplying the information.}
Italian support was added in version 1.31.\footnote{Thanks to
Edoardo Pasca for supplying the information.}

Actually, \sty{fmtcount} has two modes:
\begin{itemize}
\item a multilingual mode, in which the commands \cs{numberstring}, \cs{ordinalstring}, \cs{ordinal}, and
  their variants will be formatted in the currently selected language, as per the \cs{languagename} macro set
  by \sty{babel}, \sty{polyglossia} or suchlikes, and
\item a default mode for backward compatibility in which these commands are formatted in English irrespective
  of \cs{languagename}, and to which \sty{fmtcount} falls back when it cannot detects packages such as
  \sty{babel} or \sty{polyglossia} are loaded.
\end{itemize}

For multilingual mode, \sty{fmtcount} needs to load correctly the language definition for document dialects. To
do this use
\begin{definition}[\DescribeMacro{\FCloadlang}]
\cs{FCloadlang}\marg{dialect}
\end{definition}
in the preamble --- this will both switch on multilingual mode, and load the \meta{dialect} definition. The
\meta{dialect} should match the options passed to \sty{babel} or \sty{polyglossia}. \sty{fmtcount} currently
supports the following \meta{dialect}'s: \LanguagePackageOptionList.

If you don't use \cs{FCloadlang}, \sty{fmtcount} will attempt to detect the required dialects and call
\cs{FCloadlang} for you, but this isn't guaranteed to work.  Notably, when \cs{FCloadlang} is not used and
\sty{fmtcount} has switched on multilingual mode, but without detecting the needed dialects in the preamble,
and \sty{fmtcount} has to format a number for a dialect for which definition has not been loaded (via
\cs{FCloadlang} above), then if \sty{fmtcount} detects a definition file for this dialect it will attempt to
load it, and cause an error otherwise. This loading in body has not been tested extensively, and may may cause
problems such as spurious spaces insertion before the first formatted number, so it's best to use
\cs{FCloadlang} explicitely in the preamble.

If the French language is selected, the \texttt{french} option let you
configure the dialect and other aspects. The \texttt{abbr} also has
some influence with French. Please refer to \S~\ref
{sec:options-french}.


The male gender for all languages is used by default, however the
feminine or neuter forms can be obtained by passing \texttt{f} or
\texttt{n} as an optional argument to \cs{ordinal},
\cs{ordinalnum} etc.  For example:
\verb"\numberstring{section}[f]". Note that the optional argument
comes \emph{after} the compulsory argument.  If a gender is
not defined in a given language, the masculine version will
be used instead.

Let me know if you find any spelling mistakes (has been known
to happen in English, let alone other languages with which I'm not so
familiar.) If you want to add support for another language,
you will need to let me know how to form the numbers and ordinals 
from~0 to~99999 in that language for each gender.

\subsection{Options for setting ordinal ending position raise/level}
\label{sec:options-fmtord}

\begin{definition}[\DescribeMacro{\fmtcountsetoptions}]
\cs{fmtcountsetoptions}\verb"{"\meta{language}\verb"={fmtord="\meta{type}\verb'}}'
\end{definition}
where \meta{language} is one of the supported language \meta{type} is either \texttt{level} or \texttt{raise}
or \texttt{undefine}. If the value is \texttt{level} or \texttt{raise}, then that will set the \texttt{fmtord}
option accordingly\footnote{see~\S~\ref{sec:package-options}} only for that language \meta{language}. If the
value is \texttt{undefine}, then the non-language specific behaviour is followed.

Some \meta{language} are synonyms, here is a table:

\begin{center}
  \begin{tabular}{|l|l|}\hline
    \textbf{language}& \textbf{alias(es)}\\\hline
    english& british\\\hline
    french&frenchb\\\hline
            &germanb\\
    german&ngerman\\
            &ngermanb\\\hline
    USenglish&american\\\hline
  \end{tabular}
\end{center}

\subsection{Options for French}
\label{sec:options-french}

This section is in French, as it is most useful to French speaking people.

\selectlanguage{french} Il est possible de configurer plusieurs
aspects de la numérotation en français avec les options
\texttt{french} et \texttt{abbr}. Ces options n'ont d'effet que si le
langage \texttt{french} est chargé.

\begin{definition}[\DescribeMacro{\fmtcountsetoptions}]
\cs{fmtcountsetoptions}\verb"{french="\marg{french options}\verb'}'
\end{definition}
L'argument \meta{french options} est une liste entre accolades et
séparée par des virgules de réglages de la forme
``\meta{clef}\texttt{=}\meta{valeur}'', chacun de ces réglages est
ci-après désigné par ``option française'' pour le distinguer des
``options générales'' telles que \texttt{french}.

Le dialecte peut être sélectionné avec l'option française
\texttt{dialect} dont la valeur \meta{dialect} peut être
\texttt{france}, \texttt{belgian} ou \texttt{swiss}.
\begin{definition}[\DescribeOption{dialect}]
\cs{fmtcountsetoptions}\verb"{french={dialect="\marg{dialect}\verb'}}'
\end{definition}
\begin{definition}[\DescribeOption{french}]
\cs{fmtcountsetoptions}\verb"{french="\meta{dialect}\verb'}'
\end{definition}

Pour alléger la notation et par souci de rétro-compatibilité
\texttt{france}, \texttt{belgian} ou \texttt{swiss} sont également des
\meta{clef}s pour \meta{french options} à utiliser sans \meta{valeur}.

L'effet de l'option \texttt{dialect} est illustré ainsi:\newline
\begin{tabularx}{\linewidth}{@{}lX@{}}
  \pkgopt{france}& soixante-dix pour 70, quatre-vingts pour 80, et
  quatre-vingts-dix pour 90,\\
  \pkgopt{belgian} & septante pour 70, quatre-vingts pour 80, et
  nonante pour 90, \\
  \pkgopt{swiss} &septante pour 70, huitante\footnote{voir
    \href{http://www.alain.be/Boece/huitante_octante.html}{Octante et
      huitante} sur le site d'Alain Lassine} pour 80, et
  nonante pour 90
\end{tabularx}
Il est à noter que la variante \texttt{belgian} est parfaitement
correcte pour les francophones français\footnote{je précise que
  l'auteur de ces lignes est français}, et qu'elle est également
utilisée en Suisse Romande hormis dans les cantons de Vaud, du Valais
et de Fribourg. En ce qui concerne le mot ``octante'', il n'est
actuellement pas pris en charge et n'est guère plus utilisé, ce qui
est sans doute dommage car il est sans doute plus acceptable que le
``huitante'' de certains de nos amis suisses.

\begin{definition}[\DescribeOption{abbr}]
\cs{fmtcountsetoptions}\verb"{abbr="\meta{boolean}\verb'}'
\end{definition}
L'option générale \texttt{abbr} permet de changer l'effet de
\cs{ordinal}. Selon \meta{boolean} on a:\newline
\begin{tabularx}{\linewidth}{@{}lX@{}}
  \pkgopt{true}& pour produire des ordinaux de la forme
  {\def\languagename{french}\csname fmtord@abbrvtrue\endcsname\ordinalnum{2}} (par défaut), ou\\
  \pkgopt{false}& pour produire des ordinaux de la forme
  {\def\languagename{french}\csname fmtord@abbrvfalse\endcsname\ordinalnum{2}} \\
\end{tabularx}

\begin{definition}[\DescribeOption{vingt plural}]
\cs{fmtcountsetoptions}\verb"{french={vingt plural="\meta{french plural control}\verb'}}'
\end{definition}
\begin{definition}[\DescribeOption{cent plural}]
\cs{fmtcountsetoptions}\verb"{french={cent plural="\meta{french plural control}\verb'}}'
\end{definition}
\begin{definition}[\DescribeOption{mil plural}]
\cs{fmtcountsetoptions}\verb"{french={mil plural="\meta{french plural control}\verb'}}'
\end{definition}
\begin{definition}[\DescribeOption{n-illion plural}]
\cs{fmtcountsetoptions}\verb"{french={n-illion plural="\meta{french plural control}\verb'}}'
\end{definition}
\begin{definition}[\DescribeOption{n-illiard plural}]
\cs{fmtcountsetoptions}\verb"{french={n-illiard plural="\meta{french plural control}\verb'}}'
\end{definition}
\begin{definition}[\DescribeOption{all plural}]
\cs{fmtcountsetoptions}\verb"{french={all plural="\meta{french plural control}\verb'}}'
\end{definition}
Les options \texttt{vingt plural}, \texttt{cent plural}, \texttt{mil plural}, \texttt{n-illion plural}, et
\texttt{n-illiard plural}, permettent de contrôler très finement l'accord en nombre des mots respectivement
vingt, cent, mil, et des mots de la forme \meta{\(n\)}illion et \meta{\(n\)}illiard, où \meta{\(n\)} désigne
`m' pour 1, `b' pour 2, 'tr' pour 3, etc. L'option \texttt{all plural} est un raccourci permettant de
contrôler de concert l'accord en nombre de tous ces mots. Tous ces paramètres valent \texttt{reformed} par
défaut.

Attention, comme on va l'expliquer, seules quelques combinaisons de configurations de ces options donnent un
orthographe correcte vis à vis des règles en vigueur. La raison d'être de ces options est la suivante~:
\begin{itemize}
\item la règle de l'accord en nombre des noms de nombre dans un numéral cardinal dépend de savoir s'il a
  vraiment une valeur cardinale ou bien une valeur ordinale, ainsi on écrit \og aller à la page deux-cent
  (sans s) d'un livre de deux-cents (avec s) pages\fg, il faut donc pouvoir changer la configuration pour
  sélectionner le cas considéré,
\item un autre cas demandant quelque configurabilité est celui de \og mil\fg\ et \og mille\fg. Pour rappel \og
  mille\fg\ est le pluriel irrégulier de \og mil\fg, mais l'alternance mil/mille est rare, voire pédante, car
  aujourd'hui \og mille\fg\ n'est utilisé que comme un mot invariable, en effet le sort des pluriels étrangers
  est systématiquement de finir par disparaître comme par exemple \og scénarii\fg\ aujourd'hui supplanté par
  \og scénarios\fg. Pour continuer à pouvoir écrire \og mil\fg, il aurait fallu former le pluriel comme \og
  mils\fg, ce qui n'est pas l'usage.  Certaines personnes utilisent toutefois encore \og mil\fg\ dans les
  dates, par exemple \og mil neuf cent quatre-vingt quatre\fg\ au lieu de \og mille neuf cent quatre-vingt
  quatre\fg,
\item finalement les règles du français quoique bien définies ne sont pas très cohérentes et il est donc
  inévitable qu'un jour ou l'autre on on les simplifie. Le paquetage \styfmt{fmtcount} est déjà prêt à cette
  éventualité.
\end{itemize}

Le paramètre \meta{french plural control} peut prendre les valeurs suivantes:\newline
\settowidth{\tabcolwidth}{\pkgopt{multiple lng-width}}
\begin{supertabular}{@{}p{\tabcolwidth}p{\dimexpr\linewidth-\tabcolwidth-2\tabcolsep}@{}}
  \pkgopt{traditional}& pour sélectionner la règle en usage chez les adultes à la date de parution de ce
  document, et dans le cas des numéraux cardinaux, lorsqu'ils ont une valeur cardinale,\\
  \pkgopt{reformed}& pour suivre toute nouvelle recommandation à la date de parution de ce document, , et
  dans le cas des numéraux cardinaux, lorsqu'ils ont une valeur cardinale, l'idée des options
  \texttt{traditional} et \texttt{reformed} est donc de pouvoir contenter à la fois les anciens et les
  modernes, mais à dire vrai à la date où ce document est écrit elles ont exactement
  le même effet,\\
  \pkgopt{traditional o}& pareil que \texttt{traditional} mais dans le cas des numéraux cardinaux,
  lorsqu'ils
  ont une valeur ordinale,\\
  \pkgopt{reformed o}& pareil que \texttt{reformed} mais dans le cas des numéraux cardinaux, lorsqu'ils ont
  une valeur ordinale, de même que précédemment \texttt{reformed o} et \texttt{traditional o} ont
  exactement le même effet,\\
  \pkgopt{always}& pour marquer toujours le pluriel, ceci n'est correct que pour \og mil\fg\ vis à vis des
  règles en vigueur,\\
  \pkgopt{never}& pour ne jamais marquer le pluriel, ceci est incorrect vis à vis des règles d'orthographe
  en vigueur,\\
  \pkgopt{multiple}& pour marquer le pluriel lorsque le nombre considéré est multiplié par au moins 2, ceci
  est la règle en vigueur pour les nombres de la forme \meta{\(n\)}illion et \meta{\(n\)}illiard lorsque le
  nombre a une valeur cardinale,\\
  \pkgopt{multiple g-last}& pour marquer le pluriel lorsque le nombre considéré est multiplié par au moins 2
  est est \emph{\textbf{g}lobalement} en dernière position, où ``globalement'' signifie qu'on considère le
  nombre formaté en entier, ceci est incorrect vis à vis des règles d'orthographe
  en vigueur,\\
  \pkgopt{multiple l-last}& pour marquer le pluriel lorsque le nombre considéré est multiplié par au moins 2
  et est \emph{\textbf{l}ocalement} en dernière position, où ``localement'' siginifie qu'on considère
  seulement la portion du nombre qui multiplie soit l'unité, soit un \meta{\(n\)}illion ou un
  \meta{\(n\)}illiard~; ceci est la convention en vigueur pour le pluriel de ``vingt'' et de ``cent''
  lorsque le nombre formaté a une valeur cardinale,\\
  \pkgopt{multiple lng-last}& pour marquer le pluriel lorsque le nombre considéré est multiplié par au moins
  2 et est \emph{\textbf{l}ocalement} mais \emph{\textbf{n}on \textbf{g}lobablement} en dernière position,
  où ``localement'' et \emph{globablement} on la même siginification que pour les options \texttt{multiple
    g-last} et \texttt{multiple l-last}~; ceci est la convention en vigueur pour le pluriel de ``vingt'' et
  de ``cent'' lorsque le nombre formaté a une valeur ordinale,\\
  \pkgopt{multiple ng-last}& pour marquer le pluriel lorsque le nombre considéré est multiplié par au moins
  2, et \emph{\textbf{n}}'est pas \emph{\textbf{g}lobalement} en dernière position, où ``globalement'' a la
  même signification que pour l'option \texttt{multiple g-last}~; ceci est la règle que j'infère être en
  vigueur pour les nombres de la forme \meta{\(n\)}illion et \meta{\(n\)}illiard lorsque le nombre a une
  valeur ordinale, mais à dire vrai pour des nombres aussi grands, par exemple \og deux millions\fg, je
  pense qu'il n'est tout simplement pas d'usage de dire \og l'exemplaire deux million(s?)\fg\ pour \og le
  deux millionième
  exemplaire\fg.\\
\end{supertabular}

L'effet des paramètres \texttt{traditional}, \texttt{traditional o}, \texttt{reformed}, et \texttt{reformed
  o}, est le suivant~:

\setlength{\tabcolwidth}{\linewidth}
\addtolength{\tabcolwidth}{-10\tabcolsep}
\addtolength{\tabcolwidth}{-6\arrayrulewidth}
\setlength{\coliwidth}{0.398\tabcolwidth}
\addtolength{\coliwidth}{\arrayrulewidth}
\addtolength{\coliwidth}{2\tabcolsep}
\noindent\begin{tabular*}{\linewidth}{|%
    >{\centering\arraybackslash\ttfamily}p{\dimexpr0.204\tabcolwidth-\arrayrulewidth-\doublerulesep}||%
    *{4}{>{\centering\arraybackslash\ttfamily}p{0.199\tabcolwidth}|}}\hline
  \textrm{\meta{x} dans ``\meta{x} }plural\textrm{''}&traditional&reformed&traditional o&reformed o\\\hline
  \hline
  vingt&\multicolumn{2}{c|}{\ColIRI{multiple l-last}}&\multicolumn{2}{c|}{\ColIRI{multiple lng-last}}\\\cline{1-1}
  cent&%
  \multicolumn{2}{c|}{\ColIRII{multiple l-last}}&%
  \multicolumn{2}{c|}{\ColIRII{multiple lng-last}}\\\hline
  mil&\multicolumn{4}{c|}{always}\\\hline
  n-illion&\multicolumn{2}{c|}{\ColIRI{multiple}}&%
           \multicolumn{2}{c|}{\ColIRI{multiple ng-last}}\\\cline{1-1}
  n-illiard&%
  \multicolumn{2}{c|}{\ColIRII{multiple}}&%
  \multicolumn{2}{c|}{\ColIRII{multiple ng-last}}\\\hline
\end{tabular*}

Les configurations qui respectent les règles d'orthographe sont les suivantes~:
\begin{itemize}
\item \verb"\fmtcountsetoptions{french={all plural=reformed o}}" pour formater les numéraux cardinaux à
  valeur ordinale,
\item \verb"\fmtcountsetoptions{french={mil plural=multiple}}" pour activer l'alternance mil/mille.
\item \verb"\fmtcountsetoptions{french={all plural=reformed}}" pour revenir dans la configuration par
  défaut.
\end{itemize}

\begin{definition}[\DescribeOption{dash or space}]
\cs{fmtcountsetoptions}\verb"{french={dash or space="\meta{dash or space}\verb'}}'
\end{definition}
Avant la réforme de l'orthographe de 1990, on ne met des traits d'union qu'entre les dizaines et les unités,
et encore sauf quand le nombre \(n\) considéré est tel que \(n\mod10=1\), dans ce cas on écrit ``et un''
sans trait d'union. Après la réforme de 1990, on recommande de mettre des traits d'union de partout sauf
autour de ``mille'', ``million'' et ``milliard'', et les mots analogues comme ``billion'',
``billiard''. Cette exception a toutefois été contestée par de nombreux auteurs, et on peut aussi mettre des
traits d'union de partout.  Mettre l'option \meta{dash or space} à:\newline
\begin{tabularx}{\linewidth}{lX}
  \pkgopt{traditional}&  pour sélectionner la règle d'avant la réforme de 1990,\\
  \pkgopt{1990}&  pour suivre  la recommandation de la réforme de 1990, \\
  \pkgopt{reformed}&  pour suivre  la recommandation de la dernière
  réforme pise en charge, actuellement l'effet est le même que \textrm{1990}, ou à\\
  \pkgopt{always}&  pour mettre systématiquement des traits d'union de partout.\\
\end{tabularx}
Par défaut, l'option vaut \texttt{reformed}.


\begin{definition}[\DescribeOption{scale}]
\cs{fmtcountsetoptions}\verb"{french={scale="\meta{scale}\verb'}}'
\end{definition}
L'option \texttt{scale} permet de configurer l'écriture des grands
nombres. Mettre \meta{scale} à:\newline
\begin{tabularx}{\linewidth}{lX}
  \pkgopt{recursive}&  dans ce cas \(10^{30}\) donne mille milliards de
  milliards de milliards, pour \(10^n\), on écrit \(10^{n-9\times
    \max\{(n\div 9)-1,0\}}\) suivi de la répétition \(\max\{(n\div
  9)-1,0\}\) fois de ``de milliards''\\
  \pkgopt{long}&  \(10^{6\times n}\) donne un \meta{\(n\)}illion où
  \meta{\(n\)} est remplacé par ``bi'' pour 2, ``tri'' pour 3, etc. et
  \(10^{6\times n+3}\) donne un \meta{\(n\)}illiard avec la même
  convention pour \meta{\(n\)}. L'option \texttt{long} est correcte en
  Europe, par contre j'ignore l'usage au
  Québec.\\
  \pkgopt{short}&  \(10^{6\times n}\) donne un \meta{\(n\)}illion où
  \meta{\(n\)} est remplacé par ``bi'' pour 2, ``tri'' pour 3,
  etc. L'option \texttt{short} est incorrecte en Europe.
\end{tabularx}
Par défaut, l'option vaut \texttt{recursive}.

\begin{definition}[\DescribeOption{n-illiard upto}]
\cs{fmtcountsetoptions}\verb"{french={n-illiard upto="\meta{n-illiard upto}\verb'}}'
\end{definition}
Cette option n'a de sens que si \texttt{scale} vaut
\texttt{long}. Certaines personnes préfèrent dire ``mille
\meta{$n$}illions'' qu'un ``\meta{$n$}illiard''. Mettre l'option
\texttt{n-illiard upto} à:\newline
\begin{tabularx}{\linewidth}{lX}
  \pkgopt{infinity}&  pour que \(10^{6\times n +3}\) donne
  \meta{$n$}illiards pour tout \(n>0\),\\
  \pkgopt{infty}&  même effet que \texttt{infinity}, \\
  \(k\)&  où \(k\) est un entier quelconque strictement positif, dans
  ce cas \(10^{6\times n +3}\) donne ``mille \meta{\(n\)}illions''
  lorsque \(n>k\), et donne ``\meta{\(n\)}illiard'' sinon\\
\end{tabularx}

\begin{definition}[\DescribeOption{mil plural mark}]
\cs{fmtcountsetoptions}\verb"{french={mil plural mark="\meta{any text}\verb'}}'
\end{definition}
La valeur par défaut de cette option est \og\texttt{le}\fg. Il s'agit de la terminaison ajoutée à \og
mil\fg\ pour former le pluriel, c'est à dire \og mille\fg, cette option ne sert pas à grand chose sauf dans
l'éventualité où ce pluriel serait francisé un jour --- à dire vrai si cela se produisait une alternance
mille/milles est plus vraisemblable, car \og mille\fg\ est plus fréquent que \og mil\fg\ et que les
pluriels francisés sont formés en ajoutant \og s\fg\ à la forme la plus fréquente, par exemple \og
blini/blinis\fg, alors que \og blini\fg\ veut dire \og crêpes\fg\ (au pluriel).


\selectlanguage{english}

%\subsection{Prefixes}
%\label{sec:latin-prefixes}
%
%\begin{definition}[\DescribeMacro{\latinnumeralstring}]
%\cs{latinnumeralstring}\marg{counter}\oarg{prefix options}
%\end{definition}
%
%\begin{definition}[\DescribeMacro{\latinnumeralstringnum}]
%\cs{latinnumeralstringnum}\marg{number}\oarg{prefix options}
%\end{definition}


\section{Configuration File \texttt{fmtcount.cfg}}

You can save your preferred default settings to a file called
\texttt{fmtcount.cfg}, and place it on the \TeX\ path.  These
settings will then be loaded by the \sty{fmtcount}
package.

Note that if you are using the \sty{datetime} package,
the \texttt{datetime.cfg} configuration file will override
the \texttt{fmtcount.cfg} configuration file.
For example, if \texttt{datetime.cfg} has the line:
\begin{verbatim}
\renewcommand{\fmtord}[1]{\textsuperscript{\underline{#1}}}
\end{verbatim}
and if \texttt{fmtcount.cfg} has the line:
\begin{verbatim}
\fmtcountsetoptions{fmtord=level}
\end{verbatim}
then the former definition of \cs{fmtord} will take
precedence.

\section{LaTeX2HTML style}

The \LaTeX2HTML\ style file \texttt{fmtcount.perl} is provided.
The following limitations apply:

\begin{itemize}
\item \cs{padzeroes} only has an effect in the preamble.

\item The configuration file 
\texttt{fmtcount.cfg} is currently ignored. (This is because
I can't work out the correct code to do this.  If you
know how to do this, please let me know.)  You can however
do:
\begin{verbatim}
\usepackage{fmtcount}
\html{
%\subsection{fmtcount.sty}
% This section deals with the code for |fmtcount.sty|
%    \begin{macrocode}
\NeedsTeXFormat{LaTeX2e}
\ProvidesPackage{fmtcount}[2020/01/30 v3.06]
\RequirePackage{ifthen}
%    \end{macrocode}
% \changes{3.01}{2014/12/03}{Use \styfmt{xkeyval} instead of \styfmt{keyval}, so that we do not get in trouble
% with bracket spurious removals}
%    \begin{macrocode}
\RequirePackage{xkeyval}
\RequirePackage{etoolbox}
\RequirePackage{fcprefix}
%    \end{macrocode}
% \changes{3.00}{2014/07/3}{Add \cs{RequirePackage} for \texttt{ifxetex}}
% \changes{3.05}{2017/12/24}{Stop using \cs{ifxetex} to trigger multilingual mode. Instead only packages are
% tested for being loaded.}
% \changes{1.3}{2007/7/19}{no longer using xspace package}
%\changes{1.31}{2009/10/02}{amsgen now loaded (\cs{new@ifnextchar}
% needed)}
% Need to use \cs{new@ifnextchar} instead of \cs{@ifnextchar} in
% commands that have a final optional argument (such as \cs{gls})
% so require \sty{amsgen}.
%    \begin{macrocode}
\RequirePackage{amsgen}
%    \end{macrocode}
% These commands need to be defined before the
% configuration file is loaded.
%
% Define the macro to format the |st|, |nd|, |rd| or |th| of an 
% ordinal.
% \changes{3.01}{2014/12/3}{Make \cs{fmtord} language dependent.}
% \changes{3.01}{2014/12/3}{Substitute \cs{textsuperscript} for \cs{fc@textsuperscript}, and define
% \cs{fc@textsuperscript} as \cs{fup} when defined at beginning of document, or as \cs{textsuperscript}
% otherwise}
%\begin{macro}{\fc@orddef@ult}
%    \begin{macrocode}
\providecommand*{\fc@orddef@ult}[1]{\fc@textsuperscript{#1}}
%    \end{macrocode}
%\end{macro}
%\begin{macro}{\fc@ord@multiling}
%    \begin{macrocode}
\providecommand*{\fc@ord@multiling}[1]{%
  \ifcsundef{fc@\languagename @alias@of}{%
%    \end{macrocode}
% Not a supported language, just use the default setting:
%    \begin{macrocode}
  \fc@orddef@ult{#1}}{%
  \expandafter\let\expandafter\@tempa\csname fc@\languagename @alias@of\endcsname
  \ifcsundef{fc@ord@\@tempa}{%
%    \end{macrocode}
% Not language specfic setting, just use the default setting:
%    \begin{macrocode}
    \fc@orddef@ult{#1}}{%
%    \end{macrocode}
% Language with specific setting, use that setting:
%    \begin{macrocode}
\csname fc@ord@\@tempa\endcsname{#1}}}}
%    \end{macrocode}
%\end{macro}
%\begin{macro}{\padzeroes}
%\begin{definition}
%\cs{padzeroes}\oarg{n}
%\end{definition}
% Specifies how many digits should be displayed for commands such as
% \cs{decimal} and \cs{binary}.
%    \begin{macrocode}
\newcount\c@padzeroesN
\c@padzeroesN=1\relax
\providecommand*{\padzeroes}[1][17]{\c@padzeroesN=#1}
%    \end{macrocode}
%\end{macro}
%
%\begin{macro}{\FCloadlang}
%\changes{2.0}{2012/06/3}{new}
%\changes{2.02}{2012/10/3}{ensured catcode for @ set to `letter'
%before loading file}
%\begin{definition}
%\cs{FCloadlang}\marg{language}
%\end{definition}
% Load \styfmt{fmtcount} language file,
% \texttt{fc-}\meta{language}\texttt{.def}, unless already loaded.
% Unfortunately neither \styfmt{babel} nor \styfmt{polyglossia} keep a list of loaded
% dialects, so we can't load all the necessary def files in the
% preamble as we don't know which dialects the user requires.
% Therefore the dialect definitions get loaded when a command such
% as \cs{ordinalnum} is used, if they
% haven't already been loaded.
%    \begin{macrocode}
\newcount\fc@tmpcatcode
\def\fc@languages{}%
\def\fc@mainlang{}%
\newcommand*{\FCloadlang}[1]{%
  \@FC@iflangloaded{#1}{}%
  {%
    \fc@tmpcatcode=\catcode`\@\relax
    \catcode `\@ 11\relax
    \InputIfFileExists{fc-#1.def}%
    {%
      \ifdefempty{\fc@languages}%
      {%
        \gdef\fc@languages{#1}%
      }%
      {%
         \gappto\fc@languages{,#1}%
      }%
      \gdef\fc@mainlang{#1}%
    }%
    {}%
    \catcode `\@ \fc@tmpcatcode\relax
  }%
}
%    \end{macrocode}
%\end{macro}
%\begin{macro}{\@FC@iflangloaded}
%\changes{2.0}{2012/06/3}{new}
%\begin{definition}
%\cs{@FC@iflangloaded}\marg{language}\marg{true}\marg{false}
%\end{definition}
%If fmtcount language definition file \texttt{fc-}\meta{language}\texttt{.def} has
%been loaded, do \meta{true} otherwise do \meta{false}
%    \begin{macrocode}
\newcommand{\@FC@iflangloaded}[3]{%
  \ifcsundef{ver@fc-#1.def}{#3}{#2}%
}
%    \end{macrocode}
%\end{macro}
%\begin{macro}{\ProvidesFCLanguage}
%\changes{2.0}{2012/06/3}{new}
% Declare fmtcount language definition file. Adapted from
% \ics{ProvidesFile}.
%    \begin{macrocode}
\newcommand*{\ProvidesFCLanguage}[1]{%
  \ProvidesFile{fc-#1.def}%
}
%    \end{macrocode}
%\end{macro}
%
% We need that flag to remember that a language has been loaded via package option, so that in the end we can
% set \styfmt{fmtcount} in multiling
%    \begin{macrocode}
\newif\iffmtcount@language@option
\fmtcount@language@optionfalse
%    \end{macrocode}
%\begin{macro}{\fc@supported@language@list}
%    Declare list of supported languages, as a comma separated list. No space, no empty items. Each item is a
%    language for which fmtcount is able to load language specific definitions. \texttt{Aliases but be
%    \textit{after} their meaning, for instance `american' being an alias of `USenglish', it has to appear
%    after it in the list}. The raison d'\^etre of this list is to commonalize iteration on languages for the
%    two following purposes:
%    \begin{itemize}
%    \item loading language definition as a result of the language being used by
%      \styfmt{babel}/\styfmt{polyglossia}
%    \item loading language definition as a result of package option
%    \end{itemize}
%    These two purposes cannot be handled in the same pass, we need two different passes otherwise there would
%    be some corner cases when a package would be required --- as a result of loading language definition for
%    one language --- between a \cs{DeclareOption} and a \cs{ProcessOption} which is forbidden by \LaTeXe.
%\changes{3.00}{2014/07/3}{new}
%\changes{3.05}{2017/12/26}{Add brazilian and portuguese.}
%    \begin{macrocode}
\newcommand*\fc@supported@language@list{%
arabic,%
english,%
UKenglish,%
brazilian,%
british,%
USenglish,%
american,%
spanish,%
portuges,%
portuguese,%
french,%
frenchb,%
francais,%
german,%
germanb,%
ngerman,%
ngermanb,%
italian}
%    \end{macrocode}
%\end{macro}
%\begin{macro}{\fc@iterate@on@languages}
%\begin{definition}
%\cs{fc@iterate@on@languages}\marg{body}
%\end{definition}
% Now make some language iterator, note that for the following to work properly
% \cs{fc@supported@language@list} must not be empty. \meta{body} is  a macro that takes one argument, and
% \cs{fc@iterate@on@languages} applies it iteratively :
%\changes{3.00}{2014/07/03}{new}
%    \begin{macrocode}
\newcommand*\fc@iterate@on@languages[1]{%
  \ifx\fc@supported@language@list\@empty
%    \end{macrocode}
% That case should never happen !
%    \begin{macrocode}
    \PackageError{fmtcount}{Macro `\protect\@fc@iterate@on@languages' is empty}{You should never get here:
      Something is broken within \texttt{fmtcount}, please report the issue on
      \texttt{https://github.com/search?q=fmtcount\&ref=cmdform\&type=Issues}}%
  \else
    \let\fc@iterate@on@languages@body#1
    \expandafter\@fc@iterate@on@languages\fc@supported@language@list,\@nil,%
  \fi
}
\def\@fc@iterate@on@languages#1,{%
    {%
      \def\@tempa{#1}%
      \ifx\@tempa\@nnil
        \let\@tempa\@empty
      \else
        \def\@tempa{%
          \fc@iterate@on@languages@body{#1}%
          \@fc@iterate@on@languages
        }%
      \fi
      \expandafter
    }\@tempa
}%
%    \end{macrocode}
%\end{macro}
%\begin{macro}{\@fc@loadifbabelorpolyglossialdf}
%\begin{definition}
%\cs{@fc@loadifbabelorpolyglossialdf}\marg{language}
%\end{definition}
%Loads fmtcount language file,
%\texttt{fc-}\meta{language}\texttt{.def}, 
% if one of the following condition is met:
% \begin{itemize}
% \item \styfmt{babel} language definition file \meta{language}\texttt{.ldf} has been loaded --- conditionally
% to compilation with \texttt{latex}, not \texttt{xelatex}.
% \item \styfmt{polyglossia} language definition file \texttt{gloss-}\meta{language}\texttt{.ldf} has been
%   loaded --- conditionally to compilation with \texttt{xelatex}, not \texttt{latex}.
% \item \meta{language} option has been passed to package \styfmt{fmtcount}.
% \end{itemize}
% 
%\changes{2.03}{2012/11/03}{renamed \cs{@fc@loadifbabelldf} to
% \cs{@fc@loadifbabelorpolyglossialdf}}
%\changes{2.03}{2012/11/03}{added check for polyglossia language.}
%\changes{3.00}{2014/07/03}{use \cs{ifxetex} to discriminate between \styfmt{babel} and \styfmt{polyglossia}.}
%\changes{3.05}{2017/12/24}{Stop using \cs{ifxetex} to discriminate between \styfmt{babel} and
% \styfmt{polyglossia}, instead just test which package is loaded.}
%    \begin{macrocode}
\newcommand*\@fc@loadifbabelldf[1]{\ifcsundef{ver@#1.ldf}{}{\FCloadlang{#1}}}
\newcommand*{\@fc@loadifbabelorpolyglossialdf}[1]{}
\@ifpackageloaded{polyglossia}{%
  \def\@fc@loadifbabelorpolyglossialdf#1{\IfFileExists{gloss-#1.ldf}{\ifcsundef{#1@loaded}{}{\FCloadlang{#1}}}{}%
    \@fc@loadifbabelldf{#1}%
  }%
}{\@ifpackageloaded{babel}{%
  \let\@fc@loadifbabelorpolyglossialdf\@fc@loadifbabelldf
}{}}
%    \end{macrocode}
%\end{macro}
%
% Load appropriate language definition files:
%\changes{1.1}{2007/06/14}{added check for UKenglish,
% british and USenglish babel settings}
%\changes{2.0}{2012/06/03}{changed check for \cs{l@}\meta{language} 
% to check for \cs{date}\meta{language}}
%\changes{3.00}{2014/07/03}{use iterator rather than doing it flat on each language}
%\changes{3.01}{2014/12/03}{Define language aliases to \cs{fmtord}
%   option dependent on ``true'' language .}
%    \begin{macrocode}
\fc@iterate@on@languages\@fc@loadifbabelorpolyglossialdf
%    \end{macrocode}
% By default all languages are unique --- i.e. aliases not yet defined.
%    \begin{macrocode}
\def\fc@iterate@on@languages@body#1{%
  \expandafter\def\csname fc@#1@alias@of\endcsname{#1}}
\expandafter\@fc@iterate@on@languages\fc@supported@language@list,\@nil,%
%    \end{macrocode}
% Now define those languages that are aliases of another
% language. This is done with: \cs{@tempa}\marg{alias}\marg{language}
%    \begin{macrocode}
\def\@tempa#1#2{%
  \expandafter\def\csname fc@#1@alias@of\endcsname{#2}%
}%
\@tempa{frenchb}{french}
\@tempa{francais}{french}
\@tempa{germanb}{german}
\@tempa{ngermanb}{german}
\@tempa{ngerman}{german}
\@tempa{british}{english}
\@tempa{american}{USenglish}
%    \end{macrocode}
% Now, thanks to the aliases, we are going to define one option for each language, so that each language can
% have its own settings.
%    \begin{macrocode}
\def\fc@iterate@on@languages@body#1{%
  \define@key{fmtcount}{#1}[]{%
    \@FC@iflangloaded{#1}%
    {%
      \setkeys{fc\csname fc@#1@alias@of\endcsname}{##1}%
    }{%
      \PackageError{fmtcount}%
      {Language `#1' not defined}%
      {You need to load \ifxetex polyglossia\else babel\fi\space before loading fmtcount}%
    }%
  }%
  \ifthenelse{\equal{\csname fc@#1@alias@of\endcsname}{#1}}{%
    \define@key{fc\csname fc@#1@alias@of\endcsname}{fmtord}{%
      \ifthenelse{\equal{##1}{raise}\or\equal{##1}{level}}{%
        \expandafter\let\expandafter\@tempa\csname fc@set@ord@as@##1\endcsname
        \expandafter\@tempa\csname fc@ord@#1\endcsname
      }{%
        \ifthenelse{\equal{##1}{undefine}}{%
          \expandafter\let\csname fc@ord@#1\endcsname\undefined
        }{%
          \PackageError{fmtcount}%
          {Invalid value `##1' to fmtord key}%
          {Option `fmtord' can only take the values `level', `raise'
            or `undefine'}%
        }}
    }%
  }{%
%    \end{macrocode}
% When the language \texttt{\#1} is an alias, do the same as the language of which it is an alias:
%    \begin{macrocode}
    \expandafter\let\expandafter\@tempa\csname KV@\csname fc@#1@alias@of\endcsname @fmtord\endcsname
    \expandafter\let\csname KV@#1@fmtord\endcsname\@tempa
  }%
}
\expandafter\@fc@iterate@on@languages\fc@supported@language@list,\@nil,%
%    \end{macrocode}
%\begin{option}{fmtord}
% Key to determine how to display the ordinal
% \changes{3.01}{2014/12/03}{Apply option directly, rather than doing a border effect on \cs{fmtcount@fmtord},
% and then postprocessing depending on \cs{fmtcount@fmtord} at the end of \cs{fmtcountsetoptions}}
%    \begin{macrocode}
\def\fc@set@ord@as@level#1{%
  \def#1##1{##1}%
}
\def\fc@set@ord@as@raise#1{%
  \let#1\fc@textsuperscript
}
\define@key{fmtcount}{fmtord}{%
  \ifthenelse{\equal{#1}{level}
           \or\equal{#1}{raise}}%
  {%
    \csname fc@set@ord@as@#1\endcsname\fc@orddef@ult
    \def\fmtcount@fmtord{#1}%
  }%
  {%
    \PackageError{fmtcount}%
    {Invalid value `#1' to fmtord key}%
    {Option `fmtord' can only take the values `level' or `raise'}%
  }%
}
%    \end{macrocode}
%\end{option}
%\begin{macro}{\iffmtord@abbrv}
%  Key to determine whether the ordinal superscript should be
%  abbreviated (language dependent, currently only affects French
%  ordinals, non-abbreviated French ordinals ending --- i.e. `ier' and
%  `i\`eme' --- are considered faulty.)
%    \begin{macrocode}
\newif\iffmtord@abbrv
%    \end{macrocode}
% \changes{3.01}{2014/11/03}{Make `true' the default for option
% `abbrv', as in French this is the correct behaviour, and currently
% only French uses that}
%    \begin{macrocode}
\fmtord@abbrvtrue
\define@key{fmtcount}{abbrv}[true]{%
  \ifthenelse{\equal{#1}{true}\or\equal{#1}{false}}%
  {%
    \csname fmtord@abbrv#1\endcsname
  }%
  {%
    \PackageError{fmtcount}%
    {Invalid value `#1' to fmtord key}%
    {Option `abbrv' can only take the values `true' or
     `false'}%
  }%
}
%    \end{macrocode}
%\end{macro}
%\begin{option}{prefix}
%\changes{2.0}{2012/06/03}{new}
%    \begin{macrocode}
\define@key{fmtcount}{prefix}[scale=long]{%
  \RequirePackage{fmtprefix}%
  \fmtprefixsetoption{#1}%
}
%    \end{macrocode}
%\end{option}
%\begin{macro}{\fmtcountsetoptions}
% Define command to set options.
% \changes{3.01}{2014/12/03}{Move French specific stuff to \styfmt{french.def}.}
%    \begin{macrocode}
\def\fmtcountsetoptions{%
  \def\fmtcount@fmtord{}%
  \setkeys{fmtcount}}%
%    \end{macrocode}
%\end{macro}
% Load configuration file if it exists.  This needs to be done
% before the package options, to allow the user to override
% the settings in the configuration file.
%\changes{2.0}{2012/06/03}{Now no message if fmtcount.cfg not found}
%    \begin{macrocode}
\InputIfFileExists{fmtcount.cfg}%
{%
  \PackageInfo{fmtcount}{Using configuration file fmtcount.cfg}%
}%
{%
}
%    \end{macrocode}
%\begin{macro}{\fmtcount@loaded@by@option@lang@list}
% \changes{3.01}{2014/10/03}{Declare language option so that actual loading happens after \cs{ProcessOptions},
% and the \cs{ProcessOption} only registers the language for loading.}
%    \begin{macrocode}
\newcommand*{\fmtcount@loaded@by@option@lang@list}{}
%    \end{macrocode}
%\end{macro}
%\begin{option}{\meta{language}}
%Option \meta{language} causes language \meta{language} to be registered for loading. 
%    \begin{macrocode}
\newcommand*\@fc@declare@language@option[1]{%
  \DeclareOption{#1}{%
    \ifx\fmtcount@loaded@by@option@lang@list\@empty
       \def\fmtcount@loaded@by@option@lang@list{#1}%
    \else
       \edef\fmtcount@loaded@by@option@lang@list{\fmtcount@loaded@by@option@lang@list,#1}%
    \fi
  }}%
\fc@iterate@on@languages\@fc@declare@language@option
%    \end{macrocode}
%\end{option}
%
%\begin{option}{level}
%    \begin{macrocode}
\DeclareOption{level}{\def\fmtcount@fmtord{level}%
  \def\fc@orddef@ult#1{#1}}
%    \end{macrocode}
%\end{option}
%\begin{option}{raise}
%    \begin{macrocode}
\DeclareOption{raise}{\def\fmtcount@fmtord{raise}%
  \def\fc@orddef@ult#1{\fc@textsuperscript{#1}}}
%    \end{macrocode}
%\end{option}
% Process package options 
% \changes{3.00}{2014/07/03}{Add \cs{relax} after \cs{ProcessOptions} like shown in \texttt{clsguide.pdf}}
%    \begin{macrocode}
\ProcessOptions\relax
%    \end{macrocode}
% \changes{3.01}{2014/10/03}{Load languages that have been registered for loading by package option.}
% Now we do the loading of all languages that have been set by option to be loaded.
%    \begin{macrocode}
\ifx\fmtcount@loaded@by@option@lang@list\@empty\else
\def\fc@iterate@on@languages@body#1{%
    \@FC@iflangloaded{#1}{}{%
      \fmtcount@language@optiontrue
      \FCloadlang{#1}%
     }}
\expandafter\@fc@iterate@on@languages\fmtcount@loaded@by@option@lang@list,\@nil,%
\fi
%    \end{macrocode}
%\begin{macro}{\@FCmodulo}
%\begin{definition}
%\cs{@FCmodulo}\marg{count reg}\marg{n}
%\end{definition}
%\changes{2.04}{2014/06/03}{renamed \cs{@modulo} to \cs{@FCmodulo}}
% Sets the count register to be its value modulo \meta{n}. 
% This is used for the
% date, time, ordinal and numberstring commands. (The
% \styfmt{fmtcount} package was originally part of the 
% \sty{datetime} package.)
%    \begin{macrocode}
\newcount\@DT@modctr
\newcommand*{\@FCmodulo}[2]{%
  \@DT@modctr=#1\relax
  \divide \@DT@modctr by #2\relax
  \multiply \@DT@modctr by #2\relax
  \advance #1 by -\@DT@modctr
}
%    \end{macrocode}
%\end{macro}
% The following registers are needed by |\@ordinal| etc
%    \begin{macrocode}
\newcount\@ordinalctr
\newcount\@orgargctr
\newcount\@strctr
\newcount\@tmpstrctr
%    \end{macrocode}
%Define commands that display numbers in different bases.
% Define counters and conditionals needed.
%    \begin{macrocode}
\newif\if@DT@padzeroes
\newcount\@DT@loopN
\newcount\@DT@X
%    \end{macrocode}
%\begin{macro}{\binarynum}
% Converts a decimal number to binary, and display.
%\changes{3.02}{2016/01/08}{Made robust using \texttt{etoolbox} \cs{newrobustcmd}}
%    \begin{macrocode}
\newrobustcmd*{\@binary}[1]{%
  \@DT@padzeroestrue
  \@DT@loopN=17\relax
  \@strctr=\@DT@loopN
  \whiledo{\@strctr<\c@padzeroesN}{0\advance\@strctr by \@ne}%
  \@strctr=65536\relax
  \@DT@X=#1\relax
  \loop
    \@DT@modctr=\@DT@X
    \divide\@DT@modctr by \@strctr
    \ifthenelse{\boolean{@DT@padzeroes}
       \and \(\@DT@modctr=0\)
       \and \(\@DT@loopN>\c@padzeroesN\)}%
    {}%
    {\the\@DT@modctr}%
    \ifnum\@DT@modctr=0\else\@DT@padzeroesfalse\fi
    \multiply\@DT@modctr by \@strctr
    \advance\@DT@X by -\@DT@modctr
    \divide\@strctr by \tw@
    \advance\@DT@loopN by \m@ne
  \ifnum\@strctr>\@ne
  \repeat
  \the\@DT@X
}

\let\binarynum=\@binary
%    \end{macrocode}
%\end{macro}
%\begin{macro}{\octalnum}
% Converts a decimal number to octal, and displays.
%\changes{3.02}{2016/01/08}{Made robust using \texttt{etoolbox} \cs{newrobustcmd}}
%    \begin{macrocode}
\newrobustcmd*{\@octal}[1]{%
  \@DT@X=#1\relax
  \ifnum\@DT@X>32768
    \PackageError{fmtcount}%
    {Value of counter too large for \protect\@octal}
    {Maximum value 32768}
  \else
  \@DT@padzeroestrue
  \@DT@loopN=6\relax
  \@strctr=\@DT@loopN
  \whiledo{\@strctr<\c@padzeroesN}{0\advance\@strctr by \@ne}%
  \@strctr=32768\relax
  \loop
    \@DT@modctr=\@DT@X
    \divide\@DT@modctr by \@strctr
    \ifthenelse{\boolean{@DT@padzeroes}
       \and \(\@DT@modctr=0\)
       \and \(\@DT@loopN>\c@padzeroesN\)}%
    {}{\the\@DT@modctr}%
    \ifnum\@DT@modctr=0\else\@DT@padzeroesfalse\fi
    \multiply\@DT@modctr by \@strctr
    \advance\@DT@X by -\@DT@modctr
    \divide\@strctr by \@viiipt
    \advance\@DT@loopN by \m@ne
  \ifnum\@strctr>\@ne
  \repeat
  \the\@DT@X
  \fi
}
\let\octalnum=\@octal
%    \end{macrocode}
%\end{macro}
%\begin{macro}{\@@hexadecimal}
% Converts number from 0 to 15 into lowercase hexadecimal notation.
%    \begin{macrocode}
\newcommand*{\@@hexadecimal}[1]{%
  \ifcase#10\or1\or2\or3\or4\or5\or
  6\or7\or8\or9\or a\or b\or c\or d\or e\or f\fi
}
%    \end{macrocode}
%\end{macro}
%\begin{macro}{\hexadecimalnum}
% Converts a decimal number to a lowercase hexadecimal number, 
% and displays it.
%\changes{3.02}{2016/01/08}{Made robust using \texttt{etoolbox} \cs{newrobustcmd}}
%\changes{3.06}{2018/06/27}{Rename \cs{Hexadecimalnum} to \cs{HEXADecimalnum} and \cs{Hexadecimal} to
%    \cs{HEXADecimal} and factorize the code between \cs{HEXADecimalnum} and \cs{hexadecimalnum}.}
%    \begin{macrocode}
\newrobustcmd*{\hexadecimalnum}{\@hexadecimalengine\@@hexadecimal}
%    \end{macrocode}
%\end{macro}
%\begin{macro}{\@@Hexadecimal}
% Converts number from 0 to 15 into uppercase hexadecimal notation.
%    \begin{macrocode}
\newcommand*{\@@Hexadecimal}[1]{%
  \ifcase#10\or1\or2\or3\or4\or5\or6\or
  7\or8\or9\or A\or B\or C\or D\or E\or F\fi
}
%    \end{macrocode}
%\end{macro}
%\begin{macro}{\HEXADecimalnum}
% Uppercase hexadecimal
%\changes{3.02}{2016/01/08}{Made robust using \texttt{etoolbox} \cs{newrobustcmd}}
%\changes{3.06}{2018/06/27}{Rename \cs{Hexadecimalnum} to \cs{HEXADecimalnum} and \cs{Hexadecimal} to
%    \cs{HEXADecimal} and factorize the code between \cs{HEXADecimalnum} and \cs{hexadecimalnum}.}
%    \begin{macrocode}
\newrobustcmd*{\HEXADecimalnum}{\@hexadecimalengine\@@Hexadecimal}
\newcommand*{\@hexadecimalengine}[2]{%
  \@DT@padzeroestrue
  \@DT@loopN=\@vpt
  \@strctr=\@DT@loopN
  \whiledo{\@strctr<\c@padzeroesN}{0\advance\@strctr by \@ne}%
  \@strctr=65536\relax
  \@DT@X=#2\relax
  \loop
    \@DT@modctr=\@DT@X
    \divide\@DT@modctr by \@strctr
    \ifthenelse{\boolean{@DT@padzeroes}
      \and \(\@DT@modctr=0\)
      \and \(\@DT@loopN>\c@padzeroesN\)}
    {}{#1\@DT@modctr}%
    \ifnum\@DT@modctr=0\else\@DT@padzeroesfalse\fi
    \multiply\@DT@modctr by \@strctr
    \advance\@DT@X by -\@DT@modctr
    \divide\@strctr by 16\relax
    \advance\@DT@loopN by \m@ne
  \ifnum\@strctr>\@ne
  \repeat
  #1\@DT@X
}
\def\Hexadecimalnum{%
  \PackageWarning{fmtcount}{\string\Hexadecimalnum\space is deprecated, use \string\HEXADecimalnum\space
    instead. The \string\Hexadecimalnum\space control sequence name is confusing as it can mislead in thinking
    that only the 1st letter is upper-cased.}%
  \HEXADecimalnum}
%    \end{macrocode}
%\end{macro}
%\begin{macro}{\aaalphnum}
% Lowercase alphabetical representation (a \ldots\ z aa \ldots\ zz)
% \changes{3.02}{2016/01/08}{Made robust using \texttt{etoolbox} \cs{newrobustcmd}}
% \changes{3.04}{2017/09/16}{Code optimization, use of \cs{@ne} and \cs{m@ne}, and assign \texttt{\#1} only
% once. Factorize code with \cs{@AAAlph}.}
%    \begin{macrocode}
\newrobustcmd*{\@aaalph}{\fc@aaalph\@alph}
\newcommand*\fc@aaalph[2]{%
  \@DT@loopN=#2\relax
  \@DT@X\@DT@loopN
  \advance\@DT@loopN by \m@ne
  \divide\@DT@loopN by 26\relax
  \@DT@modctr=\@DT@loopN
  \multiply\@DT@modctr by 26\relax
  \advance\@DT@X by \m@ne
  \advance\@DT@X by -\@DT@modctr
  \advance\@DT@loopN by \@ne
  \advance\@DT@X by \@ne
  \edef\@tempa{#1\@DT@X}%
  \loop
    \@tempa
    \advance\@DT@loopN by \m@ne
  \ifnum\@DT@loopN>0
  \repeat
}

\let\aaalphnum=\@aaalph
%    \end{macrocode}
%\end{macro}
%\begin{macro}{\AAAlphnum}
% Uppercase alphabetical representation (a \ldots\ z aa \ldots\ zz)
% \changes{3.02}{2016/01/08}{Made robust using \texttt{etoolbox} \cs{newrobustcmd}}
% \changes{3.04}{2017/09/16}{Code optimization, factorize code with \@aaalph.}
%    \begin{macrocode}
\newrobustcmd*{\@AAAlph}{\fc@aaalph\@Alph}%

\let\AAAlphnum=\@AAAlph
%    \end{macrocode}
%\end{macro}
%\begin{macro}{\abalphnum}
% Lowercase alphabetical representation
%\changes{3.02}{2016/01/08}{Made robust using \texttt{etoolbox} \cs{newrobustcmd}}
% \changes{3.04}{2017/09/16}{Code factorization with \cs{@ABAlph}. Code optimization, assign \texttt{\#1} only
% once. Use \cs{@ne} and \cs{m@ne}.}
%    \begin{macrocode}
\newrobustcmd*{\@abalph}{\fc@abalph\@alph}%
\newcommand*\fc@abalph[2]{%
  \@DT@X=#2\relax
  \ifnum\@DT@X>17576\relax
    \ifx#1\@alph\def\@tempa{\@abalph}%
    \else\def\@tempa{\@ABAlph}\fi
    \PackageError{fmtcount}%
    {Value of counter too large for \expandafter\protect\@tempa}%
    {Maximum value 17576}%
  \else
    \@DT@padzeroestrue
    \@strctr=17576\relax
    \advance\@DT@X by \m@ne
    \loop
      \@DT@modctr=\@DT@X
      \divide\@DT@modctr by \@strctr
      \ifthenelse{\boolean{@DT@padzeroes}
        \and \(\@DT@modctr=1\)}%
      {}{#1\@DT@modctr}%
      \ifnum\@DT@modctr=\@ne\else\@DT@padzeroesfalse\fi
      \multiply\@DT@modctr by \@strctr
      \advance\@DT@X by -\@DT@modctr
      \divide\@strctr by 26\relax
    \ifnum\@strctr>\@ne
    \repeat
    \advance\@DT@X by \@ne
    #1\@DT@X
  \fi
}

\let\abalphnum=\@abalph
%    \end{macrocode}
%\end{macro}
%\begin{macro}{\ABAlphnum}
% Uppercase alphabetical representation
%\changes{3.02}{2016/01/08}{Made robust using \texttt{etoolbox} \cs{newrobustcmd}}
%\changes{3.04}{2017/09/16}{Code optimization, factorize code with \cs{@abalph}.}
%    \begin{macrocode}
\newrobustcmd*{\@ABAlph}{\fc@abalph\@Alph}%
\let\ABAlphnum=\@ABAlph
%    \end{macrocode}
%\end{macro}
%\begin{macro}{\@fmtc@count}
% Recursive command to count number of characters in argument.
% \cs{@strctr} should be set to zero before calling it.
%    \begin{macrocode}
\def\@fmtc@count#1#2\relax{%
  \if\relax#1%
  \else
    \advance\@strctr by 1\relax
    \@fmtc@count#2\relax
  \fi
}
%    \end{macrocode}
%\end{macro}
%\begin{macro}{\@decimal}
%\changes{1.31}{2009/10/02}{fixed unwanted space.}
% Format number as a decimal, possibly padded with zeroes in front.
%\changes{3.02}{2016/01/08}{Made robust using \texttt{etoolbox} \cs{newrobustcmd}}
%    \begin{macrocode}
\newrobustcmd*{\@decimal}[1]{%
  \@strctr=0\relax
  \expandafter\@fmtc@count\number#1\relax
  \@DT@loopN=\c@padzeroesN
  \advance\@DT@loopN by -\@strctr
  \ifnum\@DT@loopN>0\relax
    \@strctr=0\relax
    \whiledo{\@strctr < \@DT@loopN}{0\advance\@strctr by 1\relax}%
  \fi
  \number#1\relax
}

\let\decimalnum=\@decimal
%    \end{macrocode}
%\end{macro}
%\begin{macro}{\FCordinal}
%\begin{definition}
%\cs{FCordinal}\marg{number}
%\end{definition}
% This is a bit cumbersome.  Previously \cs{@ordinal}
% was defined in a similar way to \cs{abalph} etc.
% This ensured that the actual value of the counter was
% written in the new label stuff in the .aux file. However
% adding in an optional argument to determine the gender
% for multilingual compatibility messed things up somewhat.
% This was the only work around I could get to keep the
% the cross-referencing stuff working, which is why
% the optional argument comes \emph{after} the compulsory
% argument, instead of the usual manner of placing it before.
% Note however, that putting the optional argument means that
% any spaces will be ignored after the command if the optional
% argument is omitted.
% Version 1.04 changed \cs{ordinal} to \cs{FCordinal}
% to prevent it clashing with the memoir class. 
% \changes{3.02}{2015/08/01}{Suppress useless \cs{expandafter}'s and and use \cs{value} instead tweaking with
% \cs{csname}. Do not use any longer \cs{protect} as \cs{ordinalnum} is made robust.}
%    \begin{macrocode}
\newcommand{\FCordinal}[1]{%
  \ordinalnum{%
    \the\value{#1}}%
}
%    \end{macrocode}
%\end{macro}
%\begin{macro}{\ordinal}
% If \cs{ordinal} isn't defined make \cs{ordinal} a synonym
% for \cs{FCordinal} to maintain compatibility with previous
% versions.
% \changes{3.00}{2014/07/03}{Use \cs{protect}, not \cs{string} in \cs{PackageWarning} to quote macros like
% shown in \texttt{clsguide.pdf}}
%    \begin{macrocode}
\ifcsundef{ordinal}
 {\let\ordinal\FCordinal}%
 {%
   \PackageWarning{fmtcount}%
   {\protect\ordinal \space already defined use 
    \protect\FCordinal \space instead.}
 }
%    \end{macrocode}
%\end{macro}
%\begin{macro}{\ordinalnum}
% Display ordinal where value is given as a number or 
% count register instead of a counter:
%\changes{1.31}{2009/10/02}{replaced \cs{@ifnextchar} with
%\cs{new@ifnextchar}}
% \changes{3.02}{2015/08/03}{Make \cs{ordinalnum} robust.}
% \changes{3.04}{2017/09/03}{Use \texttt{etoobox}'s \cs{newrobustcmd*} instead of \LaTeX\ kernel
% \cs{DeclareRobustcommand*} in order to make \cs{ordinalnum} robust. This is preferable, e.g. w.r.t. \TeX 4ht
% compilation.}
%    \begin{macrocode}
\newrobustcmd*{\ordinalnum}[1]{%
  \new@ifnextchar[%
  {\@ordinalnum{#1}}%
  {\@ordinalnum{#1}[m]}%
}
%    \end{macrocode}
%\end{macro}
%\begin{macro}{\@ordinalnum}
% Display ordinal according to gender (neuter added in v1.1,
% \cs{xspace} added in v1.2, and removed in v1.3\footnote{I
%couldn't get it to work consistently both with and without the
%optional argument}):
%    \begin{macrocode}
\def\@ordinalnum#1[#2]{%
  {%
    \ifthenelse{\equal{#2}{f}}%
    {%
      \protect\@ordinalF{#1}{\@fc@ordstr}%
    }%
    {%
      \ifthenelse{\equal{#2}{n}}%
      {%
        \protect\@ordinalN{#1}{\@fc@ordstr}%
      }%
      {%
        \ifthenelse{\equal{#2}{m}}%
        {}%
        {%
          \PackageError{fmtcount}%
           {Invalid gender option `#2'}%
           {Available options are m, f or n}%
        }%
        \protect\@ordinalM{#1}{\@fc@ordstr}%
      }%
    }%
    \@fc@ordstr
  }%
}
%    \end{macrocode}
%\end{macro}
%\begin{macro}{\storeordinal}
% Store the ordinal (first argument
% is identifying name, second argument is a counter.)
% \changes{3.02}{2015/08/03}{Suppress useless \cs{expandafter}'s and and use \cs{value} instead tweaking with
% \cs{csname}. Do not use any longer \cs{protect} as \cs{storeordinalnum} is made robust.}
% \changes{3.02}{2016/01/08}{Expand \cs{the}\cs{value}\texttt{\{\#2\}} once before passing to \cs{storeordinalnum}}
%    \begin{macrocode}
\newcommand*{\storeordinal}[2]{%
  {%
    \toks0{\storeordinalnum{#1}}%
    \expandafter
   }\the\toks0\expandafter{%
    \the\value{#2}}%
}
%    \end{macrocode}
%\end{macro}
%\begin{macro}{\storeordinalnum}
% Store ordinal (first argument
% is identifying name, second argument is a number or
% count register.)
% \changes{3.02}{2016/01/08}{Make \cs{storeordinalnum} robust with etoolbox \cs{newrobustcmd}.}
%    \begin{macrocode}
\newrobustcmd*{\storeordinalnum}[2]{%
  \@ifnextchar[%
  {\@storeordinalnum{#1}{#2}}%
  {\@storeordinalnum{#1}{#2}[m]}%
}
%    \end{macrocode}
%\end{macro}
%\begin{macro}{\@storeordinalnum}
% Store ordinal according to gender:
%    \begin{macrocode}
\def\@storeordinalnum#1#2[#3]{%
  \ifthenelse{\equal{#3}{f}}%
  {%
    \protect\@ordinalF{#2}{\@fc@ord}
  }%
  {%
    \ifthenelse{\equal{#3}{n}}%
    {%
      \protect\@ordinalN{#2}{\@fc@ord}%
    }%
    {%
      \ifthenelse{\equal{#3}{m}}%
      {}%
      {%
        \PackageError{fmtcount}%
        {Invalid gender option `#3'}%
        {Available options are m or f}%
      }%
      \protect\@ordinalM{#2}{\@fc@ord}%
    }%
  }%
  \expandafter\let\csname @fcs@#1\endcsname\@fc@ord
}
%    \end{macrocode}
%\end{macro}
%\begin{macro}{\FMCuse}
% Get stored information:
%    \begin{macrocode}
\newcommand*{\FMCuse}[1]{\csname @fcs@#1\endcsname}
%    \end{macrocode}
%\end{macro}
%\begin{macro}{\ordinalstring}
%  Display ordinal as a string (argument is a counter)
%  \changes{3.02}{2016/01/08}{Suppress useless \cs{expandafter}'s and
%  and use \cs{value} instead tweaking with \cs{csname}. Do not use
%  any longer \cs{protect} as \cs{ordinalstringnum} is made
%  robust. Expand \cs{the}\cs{value}\texttt{\{\#1\}} once before
%  passing to \cs{ordinalstringnum}}%
% \changes{3.03}{2017/09/14}{Suppress leading \cs{expandafter}'s, they are useless as \cs{ordinalstringnum} is
% supposed to fully expand its argument, and place 3 \cs{expandafter}'s before \cs{the}\cs{value} for
% compatibily with \texttt{glossaries}.}
%    \begin{macrocode}
\newcommand*{\ordinalstring}[1]{%
  \ordinalstringnum{\expandafter\expandafter\expandafter
    \the\value{#1}}%
}
%    \end{macrocode}
%\end{macro}
%\begin{macro}{\ordinalstringnum}
% Display ordinal as a string (argument is a count register or
% number.)
%\changes{1.31}{2009/10/02}{replaced \cs{@ifnextchar} with
%\cs{new@ifnextchar}}
%\changes{1.33}{2009/10/15}{Made robust}
%\changes{3.02}{2016/01/08}{Made robust using \texttt{etoolbox} \cs{newrobustcmd}}
%    \begin{macrocode}
\newrobustcmd*{\ordinalstringnum}[1]{%
  \new@ifnextchar[%
  {\@ordinal@string{#1}}%
  {\@ordinal@string{#1}[m]}%
}
%    \end{macrocode}
%\end{macro}
%\begin{macro}{\@ordinal@string}
% Display ordinal as a string according to gender.
% \changes{3.00}{2014/07/03}{Use \cs{protect}, not \cs{string} in \cs{PackageError} to quote macros like
% shown in \texttt{clsguide.pdf}}
% \changes{3.00}{2014/07/03}{Correct detailed error message, so `n' is one of available gender options}
%    \begin{macrocode}
\def\@ordinal@string#1[#2]{%
  {%
    \ifthenelse{\equal{#2}{f}}%
    {%
      \protect\@ordinalstringF{#1}{\@fc@ordstr}%
    }%
    {%
      \ifthenelse{\equal{#2}{n}}%
      {%
        \protect\@ordinalstringN{#1}{\@fc@ordstr}%
      }%
      {%
        \ifthenelse{\equal{#2}{m}}%
        {}%
        {%
          \PackageError{fmtcount}%
          {Invalid gender option `#2' to \protect\ordinalstring}%
          {Available options are m, f or n}%
        }%
        \protect\@ordinalstringM{#1}{\@fc@ordstr}%
      }%
    }%
    \@fc@ordstr
  }%
}
%    \end{macrocode}
%\end{macro}
%\begin{macro}{\storeordinalstring}
% Store textual representation of number. First argument is 
% identifying name, second argument is the counter set to the 
% required number.
% \changes{3.02}{2016/01/08}{Suppress useless \cs{expandafter}'s and
% and use \cs{value} instead tweaking with \cs{csname}. Do not use any
% longer \cs{protect} as \cs{storeordinalstringnum} is made
% robust. Expand \cs{the}\cs{value}\texttt{\{\#2\}} once before
% passing to \cs{storeordinalstringnum}}
%    \begin{macrocode}
\newcommand*{\storeordinalstring}[2]{%
  {%
    \toks0{\storeordinalstringnum{#1}}%
    \expandafter
  }\the\toks0\expandafter{\the\value{#2}}%
}
%    \end{macrocode}
%\end{macro}
%\begin{macro}{\storeordinalstringnum}
% Store textual representation of number. First argument is 
% identifying name, second argument is a count register or number.
%\changes{3.02}{2016/01/08}{Made robust using \texttt{etoolbox} \cs{newrobustcmd}}
%    \begin{macrocode}
\newrobustcmd*{\storeordinalstringnum}[2]{%
  \@ifnextchar[%
  {\@store@ordinal@string{#1}{#2}}%
  {\@store@ordinal@string{#1}{#2}[m]}%
}
%    \end{macrocode}
%\end{macro}
%\begin{macro}{\@store@ordinal@string}
% Store textual representation of number according to gender.
% \changes{3.00}{2014/07/03}{Use \cs{protect}, not \cs{string} in \cs{PackageWarning} to quote macros like
% shown in \texttt{clsguide.pdf}}
%    \begin{macrocode}
\def\@store@ordinal@string#1#2[#3]{%
  \ifthenelse{\equal{#3}{f}}%
  {%
    \protect\@ordinalstringF{#2}{\@fc@ordstr}%
  }%
  {%
    \ifthenelse{\equal{#3}{n}}%
    {%
      \protect\@ordinalstringN{#2}{\@fc@ordstr}%
    }%
    {%
      \ifthenelse{\equal{#3}{m}}%
      {}%
      {%
        \PackageError{fmtcount}%
        {Invalid gender option `#3' to \protect\ordinalstring}%
        {Available options are m, f or n}%
      }%
      \protect\@ordinalstringM{#2}{\@fc@ordstr}%
    }%
  }%
  \expandafter\let\csname @fcs@#1\endcsname\@fc@ordstr
}
%    \end{macrocode}
%\end{macro}
%\begin{macro}{\Ordinalstring}
% Display ordinal as a string with initial letters in upper case
% (argument is a counter)
% \changes{3.02}{2016/01/08}{Suppress useless \cs{expandafter}'s and
% and use \cs{value} instead tweaking with \cs{csname}. Do not use any
% longer \cs{protect} as \cs{Ordinalstringnum} is made
% robust. Expand \cs{the}\cs{value}\texttt{\{\#1\}} once before
% passing to \cs{Ordinalstringnum}}
% \changes{3.03}{2017/09/14}{Suppress leading \cs{expandafter}'s, they are useless as \cs{Ordinalstringnum} is
% supposed to fully expand its argument, and place 3 \cs{expandafter}'s before \cs{the}\cs{value} for
% compatibily with \texttt{glossaries}.}
%    \begin{macrocode}
\newcommand*{\Ordinalstring}[1]{%
  \Ordinalstringnum{\expandafter\expandafter\expandafter\the\value{#1}}%
}
%    \end{macrocode}
%\end{macro}
%\begin{macro}{\Ordinalstringnum}
% Display ordinal as a string with initial letters in upper case
% (argument is a number or count register)
%\changes{1.31}{2009/10/02}{replaced \cs{@ifnextchar} with
%\cs{new@ifnextchar}}
%\changes{3.02}{2016/01/08}{Made robust using \texttt{etoolbox} \cs{newrobustcmd}}
%    \begin{macrocode}
\newrobustcmd*{\Ordinalstringnum}[1]{%
  \new@ifnextchar[%
  {\@Ordinal@string{#1}}%
  {\@Ordinal@string{#1}[m]}%
}
%    \end{macrocode}
%\end{macro}
%\begin{macro}{\@Ordinal@string}
% Display ordinal as a string with initial letters in upper case
% according to gender
%    \begin{macrocode}
\def\@Ordinal@string#1[#2]{%
  {%
    \ifthenelse{\equal{#2}{f}}%
    {%
      \protect\@OrdinalstringF{#1}{\@fc@ordstr}%
    }%
    {%
      \ifthenelse{\equal{#2}{n}}%
      {%
        \protect\@OrdinalstringN{#1}{\@fc@ordstr}%
      }%
      {%
        \ifthenelse{\equal{#2}{m}}%
        {}%
        {%
          \PackageError{fmtcount}%
          {Invalid gender option `#2'}%
          {Available options are m, f or n}%
        }%
        \protect\@OrdinalstringM{#1}{\@fc@ordstr}%
      }%
    }%
    \@fc@ordstr
  }%
}
%    \end{macrocode}
%\end{macro}
%\begin{macro}{\storeOrdinalstring}
% Store textual representation of number, with initial letters in 
% upper case. First argument is identifying name, second argument 
% is the counter set to the 
% required number.
% \changes{3.02}{2016/01/08}{Suppress useless \cs{expandafter}'s and
% and use \cs{value} instead tweaking with \cs{csname}. Do not use any
% longer \cs{protect} as \cs{storeOrdinalstringnum} is made
% robust. Expand \cs{the}\cs{value}\texttt{\{\#1\}} once before
% passing to \cs{storeOrdinalstringnum}}
%    \begin{macrocode}
\newcommand*{\storeOrdinalstring}[2]{%
  {%
    \toks0{\storeOrdinalstringnum{#1}}%
    \expandafter
  }\the\toks0\expandafter{\the\value{#2}}%
}
%    \end{macrocode}
%\end{macro}
%\begin{macro}{\storeOrdinalstringnum}
% Store textual representation of number, with initial letters in 
% upper case. First argument is identifying name, second argument 
% is a count register or number.
%\changes{3.02}{2016/01/08}{Made robust using \texttt{etoolbox} \cs{newrobustcmd}}
%    \begin{macrocode}
\newrobustcmd*{\storeOrdinalstringnum}[2]{%
  \@ifnextchar[%
  {\@store@Ordinal@string{#1}{#2}}%
  {\@store@Ordinal@string{#1}{#2}[m]}%
}
%    \end{macrocode}
%\end{macro}
%\begin{macro}{\@store@Ordinal@string}
% Store textual representation of number according to gender, 
% with initial letters in upper case.
%    \begin{macrocode}
\def\@store@Ordinal@string#1#2[#3]{%
  \ifthenelse{\equal{#3}{f}}%
  {%
    \protect\@OrdinalstringF{#2}{\@fc@ordstr}%
  }%
  {%
    \ifthenelse{\equal{#3}{n}}%
    {%
      \protect\@OrdinalstringN{#2}{\@fc@ordstr}%
    }%
    {%
      \ifthenelse{\equal{#3}{m}}%
      {}%
      {%
        \PackageError{fmtcount}%
        {Invalid gender option `#3'}%
        {Available options are m or f}%
      }%
      \protect\@OrdinalstringM{#2}{\@fc@ordstr}%
    }%
  }%
  \expandafter\let\csname @fcs@#1\endcsname\@fc@ordstr
}
%    \end{macrocode}
%\end{macro}
%
%\begin{macro}{\storeORDINALstring}
% Store upper case textual representation of ordinal. The first 
% argument is identifying name, the second argument is a counter.
% \changes{3.02}{2016/01/08}{Suppress useless \cs{expandafter}'s and
% and use \cs{value} instead tweaking with \cs{csname}. Do not use any
% longer \cs{protect} as \cs{storeORDINALstringnum} is made
% robust. Expand \cs{the}\cs{value}\texttt{\{\#2\}} once before
% passing to \cs{storeORDINALstringnum}.}
%    \begin{macrocode}
\newcommand*{\storeORDINALstring}[2]{%
  {%
    \toks0{\storeORDINALstringnum{#1}}%
    \expandafter
  }\the\toks0\expandafter{\the\value{#2}}%
}
%    \end{macrocode}
%\end{macro}
%\begin{macro}{\storeORDINALstringnum}
% As above, but the second argument is a count register or a
% number.
%\changes{3.02}{2016/01/08}{Made robust using \texttt{etoolbox} \cs{newrobustcmd}}
%    \begin{macrocode}
\newrobustcmd*{\storeORDINALstringnum}[2]{%
  \@ifnextchar[%
  {\@store@ORDINAL@string{#1}{#2}}%
  {\@store@ORDINAL@string{#1}{#2}[m]}%
}
%    \end{macrocode}
%\end{macro}
%\begin{macro}{\@store@ORDINAL@string}
% Gender is specified as an optional argument at the end.
%    \begin{macrocode}
\def\@store@ORDINAL@string#1#2[#3]{%
  \ifthenelse{\equal{#3}{f}}%
  {%
    \protect\@ordinalstringF{#2}{\@fc@ordstr}%
  }%
  {%
    \ifthenelse{\equal{#3}{n}}%
    {%
      \protect\@ordinalstringN{#2}{\@fc@ordstr}%
    }%
    {%
      \ifthenelse{\equal{#3}{m}}%
      {}%
      {%
        \PackageError{fmtcount}%
        {Invalid gender option `#3'}%
        {Available options are m or f}%
      }%
      \protect\@ordinalstringM{#2}{\@fc@ordstr}%
    }%
  }%
%    \end{macrocode}
% \changes{3.01}{2014/11/03}{Protect \cs{`}.}
%    \begin{macrocode}
  \expandafter\protected@edef\csname @fcs@#1\endcsname{%
    \noexpand\MakeUppercase{\@fc@ordstr}%
  }%
}
%    \end{macrocode}
%\end{macro}
%\begin{macro}{\ORDINALstring}
% Display upper case textual representation of an ordinal. The
% argument must be a counter.
% \changes{3.02}{2016/01/08}{Suppress useless \cs{expandafter}'s and
% and use \cs{value} instead tweaking with \cs{csname}. Do not use any
% longer \cs{protect} as \cs{ORDINALstringnum} is made
% robust. Expand \cs{the}\cs{value}\texttt{\{\#1\}} once before
% passing to \cs{ORDINALstringnum}}
% \changes{3.03}{2017/09/14}{Suppress leading \cs{expandafter}'s, they are useless as \cs{ORDINALstringnum} is
% supposed to fully expand its argument, and place 3 \cs{expandafter}'s before \cs{the}\cs{value} for
% compatibily with \texttt{glossaries}.}
%    \begin{macrocode}
\newcommand*{\ORDINALstring}[1]{%
  \ORDINALstringnum{\expandafter\expandafter\expandafter
    \the\value{#1}%
  }%
}
%    \end{macrocode}
%\end{macro}
%\begin{macro}{\ORDINALstringnum}
% As above, but the argument is a count register or a number.
%\changes{1.31}{2009/10/02}{replaced \cs{@ifnextchar} with
%\cs{new@ifnextchar}}
%\changes{3.02}{2016/01/08}{Made robust using \texttt{etoolbox} \cs{newrobustcmd}}
%    \begin{macrocode}
\newrobustcmd*{\ORDINALstringnum}[1]{%
  \new@ifnextchar[%
  {\@ORDINAL@string{#1}}%
  {\@ORDINAL@string{#1}[m]}%
}
%    \end{macrocode}
%\end{macro}
%\begin{macro}{\@ORDINAL@string}
% Gender is specified as an optional argument at the end.
%    \begin{macrocode}
\def\@ORDINAL@string#1[#2]{%
  {%
    \ifthenelse{\equal{#2}{f}}%
    {%
      \protect\@ordinalstringF{#1}{\@fc@ordstr}%
    }%
    {%
      \ifthenelse{\equal{#2}{n}}%
      {%
        \protect\@ordinalstringN{#1}{\@fc@ordstr}%
      }%
      {%
        \ifthenelse{\equal{#2}{m}}%
        {}%
        {%
          \PackageError{fmtcount}%
          {Invalid gender option `#2'}%
          {Available options are m, f or n}%
        }%
        \protect\@ordinalstringM{#1}{\@fc@ordstr}%
      }%
    }%
    \MakeUppercase{\@fc@ordstr}%
  }%
}
%    \end{macrocode}
%\end{macro}
%\begin{macro}{\storenumberstring}
% Convert number to textual respresentation, and store. First 
% argument is the identifying name, second argument is a counter 
% containing the number.
% \changes{3.02}{2016/01/08}{Suppress useless \cs{expandafter}'s and
% and use \cs{value} instead tweaking with \cs{csname}. Do not use any
% longer \cs{protect} as \cs{storenumberstringnum} is made
% robust. Expand \cs{the}\cs{value}\texttt{\{\#2\}} once before
% passing to \cs{storenumberstringnum}}
%    \begin{macrocode}
\newcommand*{\storenumberstring}[2]{%
  \expandafter\protect\expandafter\storenumberstringnum{#1}{%
    \expandafter\the\value{#2}}%
}
%    \end{macrocode}
%\end{macro}
%\begin{macro}{\storenumberstringnum}
% As above, but second argument is a number or count register.
%    \begin{macrocode}
\newcommand{\storenumberstringnum}[2]{%
  \@ifnextchar[%
  {\@store@number@string{#1}{#2}}%
  {\@store@number@string{#1}{#2}[m]}%
}
%    \end{macrocode}
%\end{macro}
%\begin{macro}{\@store@number@string}
% Gender is given as optional argument, \emph{at the end}.
%    \begin{macrocode}
\def\@store@number@string#1#2[#3]{%
  \ifthenelse{\equal{#3}{f}}%
  {%
    \protect\@numberstringF{#2}{\@fc@numstr}%
  }%
  {%
    \ifthenelse{\equal{#3}{n}}%
    {%
      \protect\@numberstringN{#2}{\@fc@numstr}%
    }%
    {%
      \ifthenelse{\equal{#3}{m}}%
      {}%
      {%
        \PackageError{fmtcount}
        {Invalid gender option `#3'}%
        {Available options are m, f or n}%
      }%
      \protect\@numberstringM{#2}{\@fc@numstr}%
    }%
  }%
  \expandafter\let\csname @fcs@#1\endcsname\@fc@numstr
}
%    \end{macrocode}
%\end{macro}
%\begin{macro}{\numberstring}
% Display textual representation of a number. The argument
% must be a counter.
% \changes{3.02}{2016/01/08}{Suppress useless \cs{expandafter}'s and
% and use \cs{value} instead tweaking with \cs{csname}. Do not use any
% longer \cs{protect} as \cs{numberstringnum} is made
% robust. Expand \cs{the}\cs{value}\texttt{\{\#1\}} once before
% passing to \cs{numberstringnum}}
% \changes{3.03}{2017/09/14}{Suppress leading \cs{expandafter}'s, they are useless as \cs{numberstringnum} is
% supposed to fully expand its argument, and place 3 \cs{expandafter}'s before \cs{the}\cs{value} for
% compatibily with \texttt{glossaries}.}
%    \begin{macrocode}
\newcommand*{\numberstring}[1]{%
  \numberstringnum{\expandafter\expandafter\expandafter
    \the\value{#1}}%
}
%    \end{macrocode}
%\end{macro}
%\begin{macro}{\numberstringnum}
% As above, but the argument is a count register or a number.
%\changes{1.31}{2009/10/02}{replaced \cs{@ifnextchar} with
%\cs{new@ifnextchar}}
%\changes{3.02}{2016/01/08}{Made robust using \texttt{etoolbox} \cs{newrobustcmd}}
%    \begin{macrocode}
\newrobustcmd*{\numberstringnum}[1]{%
  \new@ifnextchar[%
  {\@number@string{#1}}%
  {\@number@string{#1}[m]}%
}
%    \end{macrocode}
%\end{macro}
%
%\begin{macro}{\@number@string}
% Gender is specified as an optional argument \emph{at the end}.
%    \begin{macrocode}
\def\@number@string#1[#2]{%
  {%
    \ifthenelse{\equal{#2}{f}}%
    {%
      \protect\@numberstringF{#1}{\@fc@numstr}%
    }%
    {%
      \ifthenelse{\equal{#2}{n}}%
      {%
         \protect\@numberstringN{#1}{\@fc@numstr}%
      }%
      {%
        \ifthenelse{\equal{#2}{m}}%
        {}%
        {%
          \PackageError{fmtcount}%
          {Invalid gender option `#2'}%
          {Available options are m, f or n}%
        }%
        \protect\@numberstringM{#1}{\@fc@numstr}%
      }%
    }%
    \@fc@numstr
  }%
}
%    \end{macrocode}
%\end{macro}
%\begin{macro}{\storeNumberstring}
% Store textual representation of number. First argument is 
% identifying name, second argument is a counter.
% \changes{3.02}{2016/01/08}{Suppress useless \cs{expandafter}'s and
% and use \cs{value} instead tweaking with \cs{csname}. Do not use any
% longer \cs{protect} as \cs{storeNumberstringnum} is made
% robust. Expand \cs{the}\cs{value}\texttt{\{\#2\}} once before
% passing to \cs{storeNumberstringnum}}
%    \begin{macrocode}
\newcommand*{\storeNumberstring}[2]{%
  {%
    \toks0{\storeNumberstringnum{#1}}%
    \expandafter
  }\the\toks0\expandafter{\the\value{#2}}%
}
%    \end{macrocode}
%\end{macro}
%\begin{macro}{\storeNumberstringnum}
% As above, but second argument is a count register or number.
%    \begin{macrocode}
\newcommand{\storeNumberstringnum}[2]{%
  \@ifnextchar[%
  {\@store@Number@string{#1}{#2}}%
  {\@store@Number@string{#1}{#2}[m]}%
}
%    \end{macrocode}
%\end{macro}
%\begin{macro}{\@store@Number@string}
% Gender is specified as an optional argument \emph{at the end}:
%    \begin{macrocode}
\def\@store@Number@string#1#2[#3]{%
  \ifthenelse{\equal{#3}{f}}%
  {%
    \protect\@NumberstringF{#2}{\@fc@numstr}%
  }%
  {%
    \ifthenelse{\equal{#3}{n}}%
    {%
      \protect\@NumberstringN{#2}{\@fc@numstr}%
    }%
    {%
      \ifthenelse{\equal{#3}{m}}%
      {}%
      {%
        \PackageError{fmtcount}%
        {Invalid gender option `#3'}%
        {Available options are m, f or n}%
      }%
      \protect\@NumberstringM{#2}{\@fc@numstr}%
    }%
  }%
  \expandafter\let\csname @fcs@#1\endcsname\@fc@numstr
}
%    \end{macrocode}
%\end{macro}
%\begin{macro}{\Numberstring}
% Display textual representation of number. The argument must be
% a counter. 
% \changes{3.02}{2016/01/08}{Suppress useless \cs{expandafter}'s and
% and use \cs{value} instead tweaking with \cs{csname}. Do not use any
% longer \cs{protect} as \cs{Numberstringnum} is made
% robust. Expand \cs{the}\cs{value}\texttt{\{\#1\}} once before
% passing to \cs{Numberstringnum}}
% \changes{3.03}{2017/09/14}{Suppress leading \cs{expandafter}'s, they are useless as \cs{Numberstringnum} is
% supposed to fully expand its argument, and place 3 \cs{expandafter}'s before \cs{the}\cs{value} for
% compatibily with \texttt{glossaries}.}
%    \begin{macrocode}
\newcommand*{\Numberstring}[1]{%
  \Numberstringnum{\expandafter\expandafter\expandafter
    \the\value{#1}}%
}
%    \end{macrocode}
%\end{macro}
%\begin{macro}{\Numberstringnum}
% As above, but the argument is a count register or number.
%\changes{1.31}{2009/10/02}{replaced \cs{@ifnextchar} with
%\cs{new@ifnextchar}}
%\changes{3.02}{2016/01/08}{Made robust using \texttt{etoolbox} \cs{newrobustcmd}}
%    \begin{macrocode}
\newrobustcmd*{\Numberstringnum}[1]{%
  \new@ifnextchar[%
  {\@Number@string{#1}}%
  {\@Number@string{#1}[m]}%
}
%    \end{macrocode}
%\end{macro}
%\begin{macro}{\@Number@string}
% Gender is specified as an optional argument at the end.
%    \begin{macrocode}
\def\@Number@string#1[#2]{%
  {%
    \ifthenelse{\equal{#2}{f}}%
    {%
      \protect\@NumberstringF{#1}{\@fc@numstr}%
    }%
    {%
      \ifthenelse{\equal{#2}{n}}%
      {%
        \protect\@NumberstringN{#1}{\@fc@numstr}%
      }%
      {%
        \ifthenelse{\equal{#2}{m}}%
        {}%
        {%
          \PackageError{fmtcount}%
          {Invalid gender option `#2'}%
          {Available options are m, f or n}%
        }%
        \protect\@NumberstringM{#1}{\@fc@numstr}%
      }%
    }%
    \@fc@numstr
  }%
}
%    \end{macrocode}
%\end{macro}
%\begin{macro}{\storeNUMBERstring}
% Store upper case textual representation of number. The first 
% argument is identifying name, the second argument is a counter.
% \changes{3.02}{2016/01/08}{Suppress useless \cs{expandafter}'s and
% and use \cs{value} instead tweaking with \cs{csname}. Do not use any
% longer \cs{protect} as \cs{storeNUMBERstringnum} is made
% robust. Expand \cs{the}\cs{value}\texttt{\{\#2\}} once before
% passing to \cs{storeNUMBERstringnum}.}
%    \begin{macrocode}
\newcommand{\storeNUMBERstring}[2]{%
  {%
    \toks0{\storeNUMBERstringnum{#1}}%
    \expandafter
    }\the\toks0\expandafter{\the\value{#2}}%
}
%    \end{macrocode}
%\end{macro}
%\begin{macro}{\storeNUMBERstringnum}
% As above, but the second argument is a count register or a
% number.
%    \begin{macrocode}
\newcommand{\storeNUMBERstringnum}[2]{%
  \@ifnextchar[%
  {\@store@NUMBER@string{#1}{#2}}%
  {\@store@NUMBER@string{#1}{#2}[m]}%
}
%    \end{macrocode}
%\end{macro}
%\begin{macro}{\@store@NUMBER@string}
% Gender is specified as an optional argument at the end.
%    \begin{macrocode}
\def\@store@NUMBER@string#1#2[#3]{%
  \ifthenelse{\equal{#3}{f}}%
  {%
    \protect\@numberstringF{#2}{\@fc@numstr}%
  }%
  {%
    \ifthenelse{\equal{#3}{n}}%
    {%
      \protect\@numberstringN{#2}{\@fc@numstr}%
    }%
    {%
      \ifthenelse{\equal{#3}{m}}%
      {}%
      {%
        \PackageError{fmtcount}%
        {Invalid gender option `#3'}%
        {Available options are m or f}%
      }%
      \protect\@numberstringM{#2}{\@fc@numstr}%
    }%
  }%
  \expandafter\edef\csname @fcs@#1\endcsname{%
    \noexpand\MakeUppercase{\@fc@numstr}%
  }%
}
%    \end{macrocode}
%\end{macro}
%\begin{macro}{\NUMBERstring}
% Display upper case textual representation of a number. The
% argument must be a counter.
% \changes{3.02}{2016/01/08}{Suppress useless \cs{expandafter}'s and
% and use \cs{value} instead tweaking with \cs{csname}. Do not use any
% longer \cs{protect} as \cs{NUMBERstringnum} is made
% robust. Expand \cs{the}\cs{value}\texttt{\{\#1\}} once before
% passing to \cs{NUMBERstringnum}.}
% \changes{3.03}{2017/09/14}{Suppress leading \cs{expandafter}'s, they are useless as \cs{NUMBERstringnum} is
% supposed to fully expand its argument, and place 3 \cs{expandafter}'s before \cs{the}\cs{value} for
% compatibily with \texttt{glossaries}.}
%    \begin{macrocode}
\newcommand*{\NUMBERstring}[1]{%
  \NUMBERstringnum{\expandafter\expandafter\expandafter
    \the\value{#1}}%
}
%    \end{macrocode}
%\end{macro}
%\begin{macro}{\NUMBERstringnum}
% As above, but the argument is a count register or a number.
%\changes{1.31}{2009/10/02}{replaced \cs{@ifnextchar} with
%\cs{new@ifnextchar}}
%\changes{3.02}{2016/01/08}{Made robust using \texttt{etoolbox} \cs{newrobustcmd}}
%    \begin{macrocode}
\newrobustcmd*{\NUMBERstringnum}[1]{%
  \new@ifnextchar[%
  {\@NUMBER@string{#1}}%
  {\@NUMBER@string{#1}[m]}%
}
%    \end{macrocode}
%\end{macro}
%\begin{macro}{\@NUMBER@string}
% Gender is specified as an optional argument at the end.
%    \begin{macrocode}
\def\@NUMBER@string#1[#2]{%
  {%
    \ifthenelse{\equal{#2}{f}}%
    {%
      \protect\@numberstringF{#1}{\@fc@numstr}%
    }%
    {%
      \ifthenelse{\equal{#2}{n}}%
      {%
         \protect\@numberstringN{#1}{\@fc@numstr}%
      }%
      {%
        \ifthenelse{\equal{#2}{m}}%
        {}%
        {%
          \PackageError{fmtcount}%
          {Invalid gender option `#2'}%
          {Available options are m, f or n}%
        }%
        \protect\@numberstringM{#1}{\@fc@numstr}%
      }%
    }%
    \protect\MakeUppercase{\@fc@numstr}%
  }%
}
%    \end{macrocode}
%\end{macro}
%\begin{macro}{\binary}
% Number representations in other bases. Binary:
% \changes{3.02}{2016/01/08}{Suppress useless \cs{expandafter}'s and
% and use \cs{value} instead tweaking with \cs{csname}. Do not use any
% longer \cs{protect} as \cs{@binary} is made
% robust. Expand \cs{the}\cs{value}\texttt{\{\#1\}} once before
% passing to \cs{@binary}}
% \changes{3.03}{2017/09/14}{Suppress leading \cs{expandafter}'s, they are useless as \cs{@binary} is supposed
% to fully expand its argument, and place 3 \cs{expandafter}'s before \cs{the}\cs{value} for compatibily with
% \texttt{glossaries}.}
%    \begin{macrocode}
\providecommand*{\binary}[1]{%
  \@binary{\expandafter\expandafter\expandafter
    \the\value{#1}}%
}
%    \end{macrocode}
%\end{macro}
%\begin{macro}{\aaalph}
% Like \ics{alph}, but goes beyond 26.
% (a \ldots\ z aa \ldots zz \ldots)
% \changes{3.02}{2016/01/08}{Suppress useless \cs{expandafter}'s and
% and use \cs{value} instead tweaking with \cs{csname}. Do not use any
% longer \cs{protect} as \cs{@aaalph} is made
% robust. Expand \cs{the}\cs{value}\texttt{\{\#1\}} once before
% passing to \cs{@aaalph}.}
% \changes{3.03}{2017/09/14}{Suppress leading \cs{expandafter}'s, they are useless as \cs{@aaalph} is supposed
% to fully expand its argument, and place 3 \cs{expandafter}'s before \cs{the}\cs{value} for compatibily with
% \texttt{glossaries}.}
%    \begin{macrocode}
\providecommand*{\aaalph}[1]{%
  \@aaalph{\expandafter\expandafter\expandafter
    \the\value{#1}}%
}
%    \end{macrocode}
%\end{macro}
%\begin{macro}{\AAAlph}
% As before, but upper case.
% \changes{3.02}{2016/01/08}{Suppress useless \cs{expandafter}'s and
% and use \cs{value} instead tweaking with \cs{csname}. Do not use any
% longer \cs{protect} as \cs{@AAAlph} is made
% robust. Expand \cs{the}\cs{value}\texttt{\{\#1\}} once before
% passing to \cs{@AAAlph}.}
% \changes{3.03}{2017/09/14}{Suppress leading \cs{expandafter}'s, they are useless as \cs{@AAAlph} is supposed
% to fully expand its argument, and place 3 \cs{expandafter}'s before \cs{the}\cs{value} for compatibily with
% \texttt{glossaries}.}
%    \begin{macrocode}
\providecommand*{\AAAlph}[1]{%
  \@AAAlph{\expandafter\expandafter\expandafter
    \the\value{#1}}%
}
%    \end{macrocode}
%\end{macro}
%\begin{macro}{\abalph}
% Like \ics{alph}, but goes beyond 26. 
% (a \ldots\ z ab \ldots az \ldots)
% \changes{3.02}{2016/01/08}{Suppress useless \cs{expandafter}'s and
% and use \cs{value} instead tweaking with \cs{csname}. Do not use any
% longer \cs{protect} as \cs{@abalph} is made
% robust. Expand \cs{the}\cs{value}\texttt{\{\#1\}} once before
% passing to \cs{@abalph}.}
% \changes{3.03}{2017/09/14}{Suppress leading \cs{expandafter}'s, they are useless as \cs{@abalph} is supposed
% to fully expand its argument, and place 3 \cs{expandafter}'s before \cs{the}\cs{value} for compatibily with
% \texttt{glossaries}.}
%    \begin{macrocode}
\providecommand*{\abalph}[1]{%
  \@abalph{\expandafter\expandafter\expandafter
    \the\value{#1}}%
}
%    \end{macrocode}
%\end{macro}
%\begin{macro}{\ABAlph}
% As above, but upper case.
% \changes{3.02}{2016/01/08}{Suppress useless \cs{expandafter}'s and
% and use \cs{value} instead tweaking with \cs{csname}. Do not use any
% longer \cs{protect} as \cs{@ABAlph} is made
% robust. Expand \cs{the}\cs{value}\texttt{\{\#1\}} once before
% passing to \cs{@ABAlph}.}
% \changes{3.03}{2017/09/14}{Suppress leading \cs{expandafter}'s, they are useless as \cs{@ABAlph} is supposed
% to fully expand its argument, and place 3 \cs{expandafter}'s before \cs{the}\cs{value} for compatibily with
% \texttt{glossaries}.}
%    \begin{macrocode}
\providecommand*{\ABAlph}[1]{%
  \@ABAlph{\expandafter\expandafter\expandafter
    \the\value{#1}}%
}
%    \end{macrocode}
%\end{macro}
%\begin{macro}{\hexadecimal}
% Hexadecimal:
% \changes{3.02}{2016/01/08}{Suppress useless \cs{expandafter}'s and
% and use \cs{value} instead tweaking with \cs{csname}. Do not use any
% longer \cs{protect} as \cs{@hexadecimal} is made
% robust. Expand \cs{the}\cs{value}\texttt{\{\#1\}} once before
% passing to \cs{@hexadecimal}.}
% \changes{3.03}{2017/09/14}{Suppress leading \cs{expandafter}'s, they are useless as \cs{@hexadecimal} is
% supposed to fully expand its argument, and place 3 \cs{expandafter}'s before \cs{the}\cs{value} for
% compatibily with \texttt{glossaries}.}
%    \begin{macrocode}
\providecommand*{\hexadecimal}[1]{%
  \hexadecimalnum{\expandafter\expandafter\expandafter
    \the\value{#1}}%
}
%    \end{macrocode}
%\end{macro}
%\begin{macro}{\HEXADecimal}
% As above, but in upper case.
% \changes{3.02}{2016/01/08}{Suppress useless \cs{expandafter}'s and
% and use \cs{value} instead tweaking with \cs{csname}. Do not use any
% longer \cs{protect} as \cs{@Hexadecimal} is made
% robust. Expand \cs{the}\cs{value}\texttt{\{\#1\}} once before
% passing to \cs{@Hexadecimal}.}
% \changes{3.03}{2017/09/14}{Suppress leading \cs{expandafter}'s, they are useless as \cs{@Hexadecimal} is
% supposed to fully expand its argument, and place 3 \cs{expandafter}'s before \cs{the}\cs{value} for
% compatibily with \texttt{glossaries}.}
%    \begin{macrocode}
\providecommand*{\HEXADecimal}[1]{%
  \HEXADecimalnum{\expandafter\expandafter\expandafter
    \the\value{#1}}%
}
\newrobustcmd*\FC@Hexadecimal@warning{%
  \PackageWarning{fmtcount}{\string\Hexadecimal\space is deprecated, use \string\HEXADecimal\space
    instead. The \string\Hexadecimal\space control sequence name is confusing as it can mislead in thinking
    that only the 1st letter is upper-cased.}%
}
\def\Hexadecimal{%
  \FC@Hexadecimal@warning
  \HEXADecimal}
%    \end{macrocode}
%\end{macro}
%\begin{macro}{\octal}
% Octal:
% \changes{3.02}{2016/01/08}{Suppress useless \cs{expandafter}'s and
% and use \cs{value} instead tweaking with \cs{csname}. Do not use any
% longer \cs{protect} as \cs{@octal} is made
% robust. Expand \cs{the}\cs{value}\texttt{\{\#1\}} once before
% passing to \cs{@octal}}
% \changes{3.03}{2017/09/14}{Suppress leading \cs{expandafter}'s, they are useless as \cs{@octal} is supposed
% to fully expand its argument, and place 3 \cs{expandafter}'s before \cs{the}\cs{value} for compatibily with
% \texttt{glossaries}.}
%    \begin{macrocode}
\providecommand*{\octal}[1]{%
  \@octal{\expandafter\expandafter\expandafter
    \the\value{#1}}%
}
%    \end{macrocode}
%\end{macro}
%\begin{macro}{\decimal}
% Decimal:
% \changes{3.02}{2016/01/08}{Suppress useless \cs{expandafter}'s and
% and use \cs{value} instead tweaking with \cs{csname}. Do not use any
% longer \cs{protect} as \cs{@decimal} is made
% robust. Expand \cs{the}\cs{value}\texttt{\{\#1\}} once before
% passing to \cs{@decimal}}
% \changes{3.03}{2017/09/14}{Suppress leading \cs{expandafter}'s, they are useless as \cs{@decimal} is supposed
% to fully expand its argument, and place 3 \cs{expandafter}'s before \cs{the}\cs{value} for compatibily with
% \texttt{glossaries}.}
%    \begin{macrocode}
\providecommand*{\decimal}[1]{%
  \@decimal{\expandafter\expandafter\expandafter
    \the\value{#1}}%
}
%    \end{macrocode}
%\end{macro}
%
%\subsubsection{Multilinguage Definitions}
% Flag \cs{fc@languagemode@detected} allows to stop scanning for multilingual mode trigger conditions. It is
% initialized to \texttt{false} as no such scanning as taken place yet.
%\changes{3.05}{2017/12/22}{New flag \cs{iffc@languagemode@detected}.}
%    \begin{macrocode}
\newif\iffc@languagemode@detected
\fc@languagemode@detectedfalse
%    \end{macrocode}
%\begin{macro}{\@setdef@ultfmtcount}
% If multilingual support is provided, make \cs{@numberstring}
% etc use the correct language (if defined).
% Otherwise use English definitions. \cs{@setdef@ultfmtcount}
% sets the macros to use English.
%\changes{3.05}{2017/12/22}{Set flag \cs{iffc@languagemode@detected}.}
%    \begin{macrocode}
\def\@setdef@ultfmtcount{%
  \fc@languagemode@detectedtrue
  \ifcsundef{@ordinalMenglish}{\FCloadlang{english}}{}%
  \def\@ordinalstringM{\@ordinalstringMenglish}%
  \let\@ordinalstringF=\@ordinalstringMenglish
  \let\@ordinalstringN=\@ordinalstringMenglish
  \def\@OrdinalstringM{\@OrdinalstringMenglish}%
  \let\@OrdinalstringF=\@OrdinalstringMenglish
  \let\@OrdinalstringN=\@OrdinalstringMenglish
  \def\@numberstringM{\@numberstringMenglish}%
  \let\@numberstringF=\@numberstringMenglish
  \let\@numberstringN=\@numberstringMenglish
  \def\@NumberstringM{\@NumberstringMenglish}%
  \let\@NumberstringF=\@NumberstringMenglish
  \let\@NumberstringN=\@NumberstringMenglish
  \def\@ordinalM{\@ordinalMenglish}%
  \let\@ordinalF=\@ordinalM
  \let\@ordinalN=\@ordinalM
  \let\fmtord\fc@orddef@ult
}
%    \end{macrocode}
%\end{macro}
%
%\begin{macro}{\fc@multiling}
%\changes{2.02}{2012/10/03}{new}
%\cs{fc@multiling}\marg{name}\marg{gender}
%\changes{3.00}{2014/07/03}{Use \cs{protect}, not \cs{string} in \cs{PackageWarning} to quote macros like
% shown in \texttt{clsguide.pdf}}
%    \begin{macrocode}
\newcommand*{\fc@multiling}[2]{%
  \ifcsundef{@#1#2\languagename}%
  {% try loading it
     \FCloadlang{\languagename}%
  }%
  {%
  }%
  \ifcsundef{@#1#2\languagename}%
  {%
    \PackageWarning{fmtcount}%
    {No support for \expandafter\protect\csname #1\endcsname\space for
     language '\languagename'}%
    \ifthenelse{\equal{\languagename}{\fc@mainlang}}%
    {%
       \FCloadlang{english}%
    }%
    {%
    }%
    \ifcsdef{@#1#2\fc@mainlang}%
    {%
       \csuse{@#1#2\fc@mainlang}%
    }%
    {%
       \PackageWarningNoLine{fmtcount}%
       {No languages loaded at all! Loading english definitions}%
       \FCloadlang{english}%
       \def\fc@mainlang{english}%
       \csuse{@#1#2english}%
    }%
  }%
  {%
    \csuse{@#1#2\languagename}%
  }%
}
%    \end{macrocode}
%\end{macro}
%\begin{macro}{\@set@mulitling@fmtcount}
% This defines the number and ordinal string macros to use 
% \cs{languagename}:
%\changes{2.0}{2012/06/03}{changed errors to warnings for
%unsupported languages}
%\changes{3.05}{2017/12/22}{Set flag \cs{iffc@languagemode@detected}.}
%    \begin{macrocode}
\def\@set@mulitling@fmtcount{%
  \fc@languagemode@detectedtrue
%    \end{macrocode}
% The masculine version of \cs{numberstring}:
%    \begin{macrocode}
  \def\@numberstringM{%
    \fc@multiling{numberstring}{M}%
  }%
%    \end{macrocode}
% The feminine version of \cs{numberstring}:
%    \begin{macrocode}
  \def\@numberstringF{%
    \fc@multiling{numberstring}{F}%
  }%
%    \end{macrocode}
% The neuter version of \cs{numberstring}:
%    \begin{macrocode}
  \def\@numberstringN{%
    \fc@multiling{numberstring}{N}%
  }%
%    \end{macrocode}
% The masculine version of \cs{Numberstring}:
%    \begin{macrocode}
  \def\@NumberstringM{%
    \fc@multiling{Numberstring}{M}%
  }%
%    \end{macrocode}
% The feminine version of \cs{Numberstring}:
%    \begin{macrocode}
  \def\@NumberstringF{%
    \fc@multiling{Numberstring}{F}%
  }%
%    \end{macrocode}
% The neuter version of \cs{Numberstring}:
%    \begin{macrocode}
  \def\@NumberstringN{%
    \fc@multiling{Numberstring}{N}%
  }%
%    \end{macrocode}
% The masculine version of \cs{ordinal}:
%    \begin{macrocode}
  \def\@ordinalM{%
    \fc@multiling{ordinal}{M}%
  }%
%    \end{macrocode}
% The feminine version of \cs{ordinal}:
%    \begin{macrocode}
  \def\@ordinalF{%
    \fc@multiling{ordinal}{F}%
  }%
%    \end{macrocode}
% The neuter version of \cs{ordinal}:
%    \begin{macrocode}
  \def\@ordinalN{%
    \fc@multiling{ordinal}{N}%
  }%
%    \end{macrocode}
% The masculine version of \cs{ordinalstring}:
%    \begin{macrocode}
  \def\@ordinalstringM{%
    \fc@multiling{ordinalstring}{M}%
  }%
%    \end{macrocode}
% The feminine version of \cs{ordinalstring}:
%    \begin{macrocode}
  \def\@ordinalstringF{%
    \fc@multiling{ordinalstring}{F}%
  }%
%    \end{macrocode}
% The neuter version of \cs{ordinalstring}:
%    \begin{macrocode}
  \def\@ordinalstringN{%
    \fc@multiling{ordinalstring}{N}%
  }%
%    \end{macrocode}
% The masculine version of \cs{Ordinalstring}:
%    \begin{macrocode}
  \def\@OrdinalstringM{%
    \fc@multiling{Ordinalstring}{M}%
  }%
%    \end{macrocode}
% The feminine version of \cs{Ordinalstring}:
%    \begin{macrocode}
  \def\@OrdinalstringF{%
    \fc@multiling{Ordinalstring}{F}%
  }%
%    \end{macrocode}
% The neuter version of \cs{Ordinalstring}:
%    \begin{macrocode}
  \def\@OrdinalstringN{%
    \fc@multiling{Ordinalstring}{N}%
  }%
%    \end{macrocode}
% Make \cs{fmtord} language dependent:
% \changes{3.01}{2014/12/03}{Make \cs{fmtord} language dependent.}
%    \begin{macrocode}
  \let\fmtord\fc@ord@multiling
}
%    \end{macrocode}
%\end{macro}
%\changes{3.05}{2017/12/22}{Change all the code about detecting multiling. \cs{ifxetex} is no longer used,
% instead we loop on the list of packages of interest \styfmt{babel}, \styfmt{polyglossia}, \styfmt{mlp}, or
% \styfmt{ngerman}, and set or not \cs{iffc@languagemode@detected} flag as a result.}
% Check to see if \styfmt{babel}, \styfmt{polyglossia}, \styfmt{mlp}, or \styfmt{ngerman} packages have been
% loaded, and if yes set \styfmt{fmtcount} in multiling. First we define some \cs{fc@check@for@multiling}
% macro to do such action where \texttt{\#1} is the package name, and \texttt{\#2} is a callback.
%    \begin{macrocode}
\def\fc@check@for@multiling#1:#2\@nil{%
  \@ifpackageloaded{#1}{%
    #2\@set@mulitling@fmtcount
  }{}%
}
%    \end{macrocode}
% Now we define \cs{fc@loop@on@multiling@pkg} as an iterator to scan whether any of \styfmt{babel},
% \styfmt{polyglossia}, \styfmt{mlp}, or \styfmt{ngerman} packages has been loaded, and if so set multilingual
% mode.
%    \begin{macrocode}
\def\fc@loop@on@multiling@pkg#1,{%
  \def\@tempb{#1}%
  \ifx\@tempb\@nnil
%    \end{macrocode}
% We have reached the end of the loop, so stop here.
%    \begin{macrocode}
     \let\fc@loop@on@multiling@pkg\@empty
  \else
%    \end{macrocode}
% Make the \cs{@ifpackageloaded} test and break the loop if it was positive.
%    \begin{macrocode}
     \fc@check@for@multiling#1\@nil
     \iffc@languagemode@detected
       \def\fc@loop@on@multiling@pkg##1\@nil,{}%
     \fi
  \fi
  \fc@loop@on@multiling@pkg   
}
%    \end{macrocode}
% Now, do the loop itself, we do this at beginning of document not to constrain the order of loading
% \styfmt{fmtcount} and the multilingual package \styfmt{babel}, \styfmt{polyglossia}, etc.:
%    \begin{macrocode}
\AtBeginDocument{%
  \fc@loop@on@multiling@pkg babel:,polyglossia:,ngerman:\FCloadlang{ngerman},\@nil,
%    \end{macrocode}
% In the case that no multilingual package (such as
% \styfmt{babel}/\styfmt{polyglossia}/\styfmt{ngerman}) has been loaded, then we go to multiling
% if a language has been loaded by package option.
%    \begin{macrocode}
   \unless\iffc@languagemode@detected\iffmtcount@language@option
%    \end{macrocode}
% If the multilingual mode has not been yet activated, but a language option has been passed to
% \styfmt{fmtcount}, we should go to multilingual mode. However, first of, we do some sanity check, as this
% may help the end user understand what is wrong: we check that macro \cs{languagename} is defined, and
% activate the multilingual mode only then, and otherwise fall back to default legacy mode.
%    \begin{macrocode}
      \ifcsundef{languagename}%
      {%
         \PackageWarning{fmtcount}{%
            `\protect\languagename' is undefined, you should use a language package such as babel/polyglossia
             when loading a language via package option. Reverting to default language.
         }%
         \@setdef@ultfmtcount
      }{%
        \@set@mulitling@fmtcount
        
%    \end{macrocode}
% Now, some more checking, having activated multilingual mode after a language option has been passed to
% \styfmt{fmtcount}, we check that the \styfmt{fmtcount} language definitions corresponding to
% \cs{languagename} have been loaded, and otherwise fall \cs{languagename} back to the latest
% \styfmt{fmtcount} language definition loaded.
%    \begin{macrocode}
        \@FC@iflangloaded{\languagename}{}{%
%    \end{macrocode}
% The current \cs{languagename} is not a \styfmt{fmtcount} language that has been previously loaded. The
% correction is to have \cs{languagename} let to \cs{fc@mainlang}. Please note that, as
% \cs{iffmtcount@language@option} is true, we know that \texttt{fmtcount} has loaded some language.
%    \begin{macrocode}
            \PackageWarning{fmtcount}{%
               Setting `\protect\languagename' to `\fc@mainlang'.\MessageBreak
               Reason is that `\protect\languagename' was `\languagename',\MessageBreak
               but `\languagename' was not loaded by fmtcount,\MessageBreak
               whereas `\fc@mainlang' was the last language loaded by fmtcount ;
            }%
            \let\languagename\fc@mainlang
         }%
      }%
   \else
       \@setdef@ultfmtcount
   \fi\fi
}
%    \end{macrocode}
% \changes{3.01}{2014/12/03}{Substitute \cs{textsuperscript} for \cs{fc@textsuperscript}, and define
% \cs{fc@textsuperscript} as \cs{fup} when defined at beginning of document, or as \cs{textsuperscript}
% otherwise}
%    \begin{macrocode}
\AtBeginDocument{%
   \ifcsundef{FBsupR}{\let\fc@textsuperscript\textsuperscript}{\let\fc@textsuperscript\fup}%
}
%    \end{macrocode}
% Backwards compatibility:
%    \begin{macrocode}
\let\@ordinal=\@ordinalM
\let\@ordinalstring=\@ordinalstringM
\let\@Ordinalstring=\@OrdinalstringM
\let\@numberstring=\@numberstringM
\let\@Numberstring=\@NumberstringM
%    \end{macrocode}
\iffalse Local variables: \fi
\iffalse mode: docTeX     \fi
\iffalse End:             \fi
}
\end{verbatim}
This, I agree, is an unpleasant cludge.

\end{itemize}

\section{Miscellaneous}
\label{sec:misc}

\subsection{Handling of spaces with tailing optional argument}
\label{sec:tailing-oarg}

Quite some of the commands in \styfmt{fmtcount} have a tailing optional argument, notably a \oarg{gender}
argument, which is due to historical reasons, and is a little unfortunate.

When the tailing optional argument is omitted, then any subsequent space will:
\begin{itemize}
\item \texttt{not} be gobbled if the command make some typset output, like \cs{ordinal} or \cs{numbestring}, and
\item be gobbled if the command stores a number into a label like \cs{storeordinalnum} or
  \cs{storenumberstring}, or make some other border effect like \cs{padzeroes} without any typeset output.
\end{itemize}

So (where we use visible spaces ``\verb*! !'' to demonstrate the point):
\begin{itemize}
\item ``x\cs{odinalnum\{2\}}\verb*! !\texttt{x}'' will be typeset to ``x\ordinalnum{2}\textvisiblespace x'', while
\item ``x\cs{storeodinalnum\{mylabel\}\{2\}}\verb*! !\texttt{x}'' will be typeset to ``xx''.
\end{itemize}

The reason for this design choice is that the commands like like \cs{ordinal} or \cs{numbestring} are usually
inserted in the flow of text, and one usually does not want subsequent spaces gobbled, while the commands like
\cs{storeordinalnum} or \cs{storenumberstring} usually stands on their own line, and one usually does not want
the tailing end-of-line to produce an extra-space.

\subsection{Macro naming conventions}
\label{sec:macro-naming}

Macros that refer to upper-casing have upper case only in the main part of their name. That is to say the
words ``store'', ``string'' or ``num'' are not upper-cased for instance in \cs{storeORDINALstringnum},
\cs{storeOrdinalstringnum} or in \cs{NUMBERstringnum}.

Furthermore, when upper-casing all the number letters is considered, the main part of the name is:
\begin{itemize}
\item all in upper-case when it consist of a single word that is not composed of a prefix+radix, for instance
  ``ORDINAL'' or ``NUMBER'', and
\item with the prefix all in upper-case, and only the first letter of the radix in upper-case for words that
  consist of a prefix+radix, for instance ``HEXADecimal'' or ``AAAlph'' because they can be considered as a
  prefix+radix construct ``hexa+decimal'' or ``aa+alph''.
\end{itemize}

Observance of this rule is the reason why macros \cs{Hexadecimal} and \cs{Hexadecimalnum} were respectively
renamed as \cs{HEXADecimal} and \cs{HEXADecimalnum} from v3.06.


\section{Acknowledgements}

I would like to thank all the people who have provided translations and made bug reports.

\section{Troubleshooting}

There is a FAQ available at: 
\url{http://theoval.cmp.uea.ac.uk/~nlct/latex/packages/faq/}.

Bug reporting should be done via the Github issue manager at: 
\url{https://github.com/nlct/fmtcount/issues/}.

\end{document}
% Local Variables:
% coding: utf-8
% compile-command: "make -C ../dist fmtcount.pdf"
% End:
