 % This file is public domain
\documentclass[a4paper]{article}

\usepackage{fmtcount}
% FCloadlang
\newcounter{N}

\padzeroes[5]

\newcommand{\printrow}[1]{%
  \newline
  \setcounter{N}{#1}
  \makebox[0.75in][r]{\decimal{N},\quad}
  \makebox[0.75in][r]{\aaalph{N},\quad}
  \makebox[0.75in][r]{\ifnum\value{N}<17577 \abalph{N}\else N/A\fi,\quad}
  \makebox[2in][l]{\ifnum\value{N}<32769\octal{N}\else N/A\fi}
}

%\renewcommand{\thesection}{\ordinal{section}}

\begin{document}

\setcounter{N}{18}
\tableofcontents
\section{Displaying the value of a counter}
\label{ex}

\begin{ttfamily}
\noindent
\makebox[0.75in][c]{decimal}
\makebox[0.75in][c]{aaalph}
\makebox[0.75in][c]{abalph}
octal
\printrow{0}
\printrow{1}
\printrow{2}
\printrow{3}
\printrow{4}
\printrow{5}
\printrow{6}
\printrow{7}
\printrow{8}
\printrow{9}
\printrow{10}
\printrow{11}
\printrow{12}
\printrow{13}
\printrow{14}
\printrow{15}
\printrow{16}
\printrow{17}
\printrow{18}
\printrow{19}
\printrow{20}
\printrow{21}
\printrow{22}
\printrow{23}
\printrow{24}
\printrow{25}
\printrow{30}
\printrow{40}
\printrow{50}
\printrow{60}
\printrow{70}
\printrow{71}
\printrow{75}
\printrow{80}
\printrow{81}
\printrow{85}
\printrow{90}
\printrow{91}
\printrow{95}
\printrow{100}
\printrow{101}
\printrow{110}
\printrow{125}
\printrow{150}
\printrow{170}
\printrow{180}
\printrow{190}
\printrow{200}
\printrow{250}
\printrow{300}
\printrow{400}
\printrow{500}
\printrow{600}
\printrow{700}
\printrow{800}
\printrow{900}
\printrow{1000}
\printrow{17576}
\printrow{32768}
\printrow{99999}
\end{ttfamily}

\section{Displaying the value of a counter --- upercase}
\label{ex}

\begin{ttfamily}
\noindent
\makebox[0.75in][c]{hexadecimal}
\makebox[0.75in][c]{Hexadecimal}
\makebox[0.75in][c]{AAAlph}
\makebox[0.75in][c]{ABAlph}
\renewcommand{\printrow}[1]{%
  \newline
  \setcounter{N}{#1}
  \makebox[0.75in][r]{\hexadecimal{N},\quad}
  \makebox[0.75in][r]{\Hexadecimal{N},\quad}
  \makebox[0.75in][r]{\AAAlph{N},\quad}
  \makebox[2in][l]{\ifnum\value{N}<17577\ABAlph{N}\else N/A\fi}
}
\printrow{0}
\printrow{1}
\printrow{2}
\printrow{3}
\printrow{4}
\printrow{5}
\printrow{6}
\printrow{7}
\printrow{8}
\printrow{9}
\printrow{10}
\printrow{11}
\printrow{12}
\printrow{13}
\printrow{14}
\printrow{15}
\printrow{16}
\printrow{17}
\printrow{18}
\printrow{19}
\printrow{20}
\printrow{21}
\printrow{22}
\printrow{23}
\printrow{24}
\printrow{25}
\printrow{30}
\printrow{40}
\printrow{50}
\printrow{60}
\printrow{70}
\printrow{71}
\printrow{75}
\printrow{80}
\printrow{81}
\printrow{85}
\printrow{90}
\printrow{91}
\printrow{95}
\printrow{100}
\printrow{101}
\printrow{110}
\printrow{125}
\printrow{150}
\printrow{170}
\printrow{180}
\printrow{190}
\printrow{200}
\printrow{250}
\printrow{300}
\printrow{400}
\printrow{500}
\printrow{600}
\printrow{700}
\printrow{800}
\printrow{900}
\printrow{1000}
\printrow{17576}
\printrow{32768}
\printrow{99999}
\end{ttfamily}

\section{Cross-Referencing}

Referencing a label: \ref{ex}.

Passing numbers explicitly (28):
\aaalphnum{28},
\AAAlphnum{28},
\abalphnum{28},
\ABAlphnum{28},
\octalnum{28},
\hexadecimalnum{28},
\Hexadecimalnum{28}.

\setcounter{N}{777}
\section{\textbackslash FCordinal in a title: \FCordinal{N}}
\section{\textbackslash octal in a title: \octal{N}}
\section{\textbackslash aaalph in a title: \aaalph{N}}
\section{\textbackslash abalph in a title: \abalph{N}}

\end{document}
